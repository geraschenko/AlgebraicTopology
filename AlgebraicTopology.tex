\documentclass[12pt]{article}

 \usepackage{amsthm}
 \usepackage{amsmath}
 \usepackage{amssymb}
 \usepackage{latexsym}
 \usepackage[all]{xy}
    \SelectTips{cm}{10} % Use the nice arrow heads
    \everyxy={<2.5em,0em>:} % Sets the scale I like
 \usepackage{accents}
 \usepackage{pstricks}
 \usepackage[paperheight=11in,%
          paperwidth=8.5in,%
          outer=1.2in,%
          inner=1in,%
          bottom=.7in,%
          top=.7in,%
          includeheadfoot]{geometry}

 \makeatletter \@addtoreset{equation}{section} \makeatother

 \theoremstyle{plain}
 \newtheorem{theorem}[equation]{Theorem}
 \newtheorem*{claim}{Claim}
 \newtheorem{lemma}[equation]{Lemma}
 \newtheorem{corollary}[equation]{Corollary}
 \newtheorem{proposition}[equation]{Proposition}

 \theoremstyle{definition}
 \newtheorem{definition}[equation]{Definition}
 \makeatletter
 \newenvironment{example}[1][]{%
    \begin{trivlist} \item[]%
    \refstepcounter{equation}\textbf{Example~\theequation}%
    \@ifnotempty{#1}{\the\thm@notefont \ (#1)}\textbf{.} }%
    {\hspace*{\fill}$\bullet$ \end{trivlist}}
 \makeatother
 \newtheorem{exercise}{$\blacktriangleright$
                       \hypertarget{Ex\theexercise}{Exercise}}[section]

 \newenvironment{solution}{%\small%
        \begin{trivlist} \item \textit{Solution}. }{%
            \hspace*{\fill} $\blacksquare$\end{trivlist}}%

 \theoremstyle{remark}
 \newtheorem{remark}[equation]{Remark}

 \renewcommand{\theequation}{\thesection.\arabic{equation}}

%%%%%%%%%%%%%%%% Anton's Shortcuts %%%%%%%%%%%%%%%%%%%%%%%%%%%%%%
% \def\proclaim #1. #2\par{\medbreak \noindent \textbf{#1.} \textit{#2}\par\medbreak}
 \newcommand{\anton}[1]{[[\index{"!@notes and corrections}
                        \ensuremath{\bigstar\bigstar\bigstar} #1]]}
 \newcommand{\bbar}[1]{\overline{#1}}
 \newcommand{\bigast}{\text{\Large $\ast$}}
 \newcommand\C{\mathcal{C}}
 \DeclareMathOperator{\coker}{coker}
 \newcommand{\CC}{\ensuremath{\mathbb{C}}}
 \newcommand{\D}{\ensuremath{\mathcal{D}}}
 \DeclareMathOperator{\End}{End}
 \DeclareMathOperator{\ext}{Ext}
 \newcommand{\F}{\mathcal{F}}
 \let\hom\relax % kills the old hom
 \DeclareMathOperator{\hom}{Hom}
 \renewcommand{\H}{\widetilde H}
 \renewcommand{\labelitemi}{--}                    % changes the default bullet in itemize
 \newcommand{\id}{\mathrm{Id}}
 \DeclareMathOperator{\im}{im}
 \newcommand{\inn}[1]{\accentset{\circ}{#1}}
 \newcommand{\po}{\rule{5pt}{.4pt}\rule{.4pt}{5.4pt}\llap{$\cdot$\hspace{1pt}}}
 \newcommand{\pb}{\rule{.4pt}{5.4pt}\rule[5pt]{5pt}{.4pt}\llap{$\cdot$\hspace{1pt}}}
 \newcommand{\PP}{\mathbb{P}}
 \newcommand{\QQ}{\ensuremath{\mathbb{Q}}}
 \newcommand{\RR}{\ensuremath{\mathbb{R}}}
 \newcommand{\smaltrix}[4]{\ensuremath{\left( %
            \begin{smallmatrix} #1 & #2 \\ #3 & #4 \end{smallmatrix} \right)}}
 \DeclareMathOperator{\tor}{Tor}
 \newcommand{\W}{\Omega}
 \newcommand{\ZZ}{\ensuremath{\mathbb{Z}}}
%%%%%%%%%%%%% End Anton's shortcuts %%%%%%%%%%%%%%%%%%%%%%%%%%%%%%

\begin{document}
 \title{\vspace*{-4\baselineskip} Anton Geraschenko's Topology Notes}
 \date{}
 \author{}
 \maketitle
 \setcounter{section}{1}
 \vspace*{-3\baselineskip}

 \section*{About these notes}
 I took Math 215A, Algebraic Topology, with Peter Teichner in Fall 2006. I took some
 notes in class and then wrote them up after class. At first, I would do one lecture at a
 time (so the earlier sections really are one lecture's worth of material). Later in the
 course, I would wait a bit before writing up a lecture because subsequent lectures often
 added context. Thus, the later sections are not in bijection with lectures, but are
 partitioned by topic. I often added proofs for statements that were not proven in class,
 or modified the proofs from class.

 These notes are not self-contained. There is often dependence on results from the
 homework and occasionally some statements are not proven at all.


 All spaces are Hausdorff and all maps are continuous.
 \section{A Theorem from Point-Set topology}
 \begin{theorem}\label{T:cmptHausbij}
   If $X$ is compact (and Hausdorff) and $Y$ is Hausdorff, then $f:X\to Y$ is closed. In
   particular, if $f$ is a bijection, then it is a homeomorphism.
 \end{theorem}
 \begin{proof}
   If $A\subseteq X$ be closed, then it is compact because closed subsets of compact sets
   are compact. Since the continuous image of a compact set is compact, $f(A)$ is
   compact. Since compact subsets of Hausdorff spaces are closed, $f(A)$ is closed.
 \end{proof}

 \section{homotopic maps}

 \begin{definition}
   The $n$-dimensional ball is $D^n = \{x\in \RR^n | \|x\|_2\le 1\}$. The sphere is
   $S^{n-1} = \partial D^n = D^n\smallsetminus \inn D^n$.
 \end{definition}
 \begin{theorem}
   If $C\subseteq \RR^n$ is convex, compact, and $\inn C\neq \varnothing$, then $C\approx
   D^n$ and $\partial C\approx S^{n-1}$.
 \end{theorem}
 \begin{proof}
   Assume $0\in \inn C$. Define $f:\partial C\to S^{n-1}$ by $f(x) = \frac{x}{\|x\|_2}$.
   $f$ is surjective because for every $v\in S^{n-1}$, there is a maximal $t\in \RR_{>0}$
   so that $tv\in \partial C$. By convexity of $C$, this $t$ is unique (here you use that
   $\inn C$ contains a ball around $0$), so $f$ is injective. Since $\partial C$ is
   compact and $S^{n-1}$ is Hausdorff, $f$ is a homeomorphism by Theorem
   \ref{T:cmptHausbij}. Similarly, extend $f$ to a homeomorphism $\tilde f:C\to D^n$.
 \end{proof}
 \begin{corollary}
   For any norm, the ball you get is homeomorphic to the ball with the 2-norm.
 \end{corollary}
 \begin{definition}
   $f_0,f_1:X\to Y$ are \emph{homotopic} (written $f_0 \simeq f_1$) if there is a
   continuous map $F:X\times I \to Y$ such that $F(x,i)=f_i(x)$ for $i=0,1$.
 \end{definition}
 If $F(x,t)$ is continuous in $t$, there is an associated map $F^*:I\to
 Y^X:=\hom_\text{Top}(X,Y)$. In this case, a homotopy of maps is just a path in the
 mapping space $Y^X$.

 We'll show next time that for $F$ to be continuous in $t$, you need $X$ to be
 \emph{locally compact} (every point must have a compact neighborhood\footnote{Since all
 our spaces are Hausdorff, this is equivalent to saying that every neighborhood of a
 point contains a compact neighborhood.}). In this case, there is a topology on $Y^X$
 which makes $F^*$ continuous.
 \begin{definition}
   The \emph{compact-open} topology on $Y^X$ has a sub-basis given by sets of the form
   $M(K,U)=\{g:X\to Y|g(K)\subseteq U\}$, where $K\subseteq X$ is compact and $U\subseteq
   Y$ is open.
 \end{definition}

 \section{Compact-Open topology and Exponential laws}
 If $X$ is compact and $Y$ is a metric space, then the compact-open topology is given by
 the metric $d(f,g)=\sup \{d(f(x),g(x)\}$.
 \begin{lemma}\label{L:locmptEvcont}
   If $X$ is locally compact, then the evaluation map $ev:Y^X\times X\to Y$, given by
   $(f,x)\mapsto f(x)$, is continuous.
 \end{lemma}
 \begin{proof}
   If $U$ is a neighborhood of $f(x)$, $f^{-1}(U)$ contains some compact open
   neighborhood $K$ of $x$. Then $ev\bigl(M(K,U)\times K\bigr)\subseteq U$ by definition,
   so $M(K,U)\times \inn K$ is an open neighborhood of $(f,x)$ in $ev^{-1}(U)$.
 \end{proof}
 \begin{theorem}
   Let $X$ be locally compact. Then $f:X\times T\to Y$ is continuous if and only if (1)
   each $f_t:X\to Y$ is continuous, and (2) $\hat f:T\to Y^X$ is continuous.
 \end{theorem}\vspace*{-\baselineskip}
 \begin{proof}
   ($\Leftarrow$) $f$ is the composition $X\times T\xrightarrow{\smaltrix 0{\hat
   f}{\id_X}0} Y^X\times X \xrightarrow{ev} Y$. By $(1)$, $\hat f(t)=f_t$ is
   in $Y^X$; $\smaltrix 0{\hat f}{\id_X}0$ is continuous by $(2)$ and $ev$ is
   continuous by Lemma \ref{L:locmptEvcont}.

   ($\Rightarrow$) $f_t$ is the composition of continuous functions
   $X\xrightarrow{x\mapsto (x,t)} X\times T\xrightarrow{f} Y$, proving $(1)$. To prove
   $(2)$, we need to prove that $\hat f^{-1}\bigl(M(K,U)\bigr)=\bigl\{t\in
   T|f(K\times\{t\})\subseteq U\bigr\}\subseteq T$ is open. Fix some $t\in T$ such that
   $f(K\times \{t\})\subseteq U$.
   \begin{claim}
     There are open sets $W\subseteq T$ and $V\subseteq X$ such that $K\times
     \{t\}\subseteq V\times W\subseteq f^{-1}(U)$
   \end{claim}
   By continuity of $f$, for each $x\in K$, there are open neighborhoods $x\in
   V_x\subseteq X$ and $t\in W_x\subseteq T$ such that $V_x\times W_x\subseteq
   f^{-1}(U)$. By compactness, $K\subseteq V_{x_1}\cup \cdots \cup V_{x_n}=:V$. Set
   $W=W_{x_1}\cap \cdots\cap W_{x_n}$. This $V$ and $W$ satisfy the claim.
   \[\begin{pspicture}(-.7,-.6)(5.5,5)
     \psframe[framearc=.1](1,1)(3,3)
       \psline{(-)}(1,-.5)(3,-.5) \rput(1,-.1){$V_{x_1}$}
       \psline{(-)}(-.5,1)(-.5,3) \rput(0,1.2){$W_{x_1}$}
     \psframe[framearc=.1](1.8,2.2)(5,4)
       \psline{(-)}(1.8,0)(5,0) \rput(4.5,-.3){$V_{x_2}$}
       \psline{(-)}(0,2.2)(0,4) \rput(-.5,3.5){$W_{x_2}$}
     \psframe[framearc=.1,linestyle=dotted](1.1,2.3)(4.9,2.9)
     \psline[linewidth=1.5pt,linecolor=darkgray](1.3,2.5)(4.7,2.5)
       \rput(4,2.7){\scriptsize $K\times \{t\}$}
     \psccurve(.8,.8)(3,.8)(3.5,1.7)(5,2)(5,4.2)(1.8,4.2)(1.5,3.5)(.8,3)
       \rput(4.3,1.3){$f^{-1}(U)$}
   \end{pspicture}\]
   Now the given $W$ is a neighborhood of $t$ in $\hat f^{-1}\bigl(M(K,U)\bigr)$.
 \end{proof}
 \begin{theorem}[Exponential Law]
   If $X$ is locally compact, then $Y^{X\times T}\approx (Y^X)^T$, $f\mapsto \hat f$.
 \end{theorem}
 \begin{proof}
   By the previous theorem, this map is bijective. It is an exercise to show the
   homeomorphism. Or look at page 531 of Hatcher.
 \end{proof}
 \begin{lemma}
   If $X$ and $T$ are locally compact, then
   \begin{itemize}
     \item $Y^X\times W^X \approx (Y\times W)^X$
     \item $Y^{X\coprod T}\approx Y^X\times Y^T$
     \item $\bigl((Y,y_0)^{(X,x_0)},\mathrm{const}_{y_0}\bigr)^{(T,t_0)}\approx (Y,y_0)^{(X\times T,X\vee
     T)} \approx (Y,y_0)^{X\wedge T}$.
   \end{itemize}
   where $X\wedge T:= X\times T/X\vee T$.
   \begin{definition}
     $f:X\to Y$ is a \emph{homotopy equivalence} if there is a $g:Y\to X$ so that $g\circ
     f\simeq \id_Y$ and $f\circ g\simeq \id_X$.
   \end{definition}
 \end{lemma}

 \section{CW complexes}
 \begin{definition}
   A space $X$ is \emph{homogeneous} if the group of homeomorphisms $X\to X$ acts
   transitively on the points of $X$.
 \end{definition}
 \begin{theorem}
   Every connected manifold is homogeneous. Every topological group is homogeneous.
 \end{theorem}
 For example, the Cantor set is homogeneous.

 The push-out and pull-back are (respectively)
 \[\xymatrix{
   A \ar[r]^{f_1} \ar[d]_{f_2} \ar@{}[dr]|(.75)\po & B_1 \ar[d]^{g_1} \ar[dr] \\
   B_2 \ar[r]^{g_2} \ar@/_2ex/[rr] & PO \ar@{-->}[r]^{\exists !} & Y
 }
 \xymatrix{
   Y \ar@{-->}[r]_{\exists !} \ar@/^2ex/[rr] \ar[dr]
   & PB\ar[r]_{g_1} \ar[d]^{g_2} \ar@{}[dr]|(.25)\pb & B_1 \ar[d]^{f_1} \\
   & B_2 \ar[r]^{f_2} & A }
 \]
 Pull-backs and push-outs exist in the category of topological spaces (the constructions
 from Set work). $PO = B_1\coprod B_2/f_1(a)\sim f_2(a)$ and $PB = \{(x,y)\in B_1\times
 B_2| f_1(x)=f_2(y)\}$.

 A \emph{CW complex} is where you attach cells in order of
 dimension. That is, define $X^{(-1)}=\varnothing$, and define the \emph{$n$-skeleton}
 $X^{(n)}$ by
 \[\xymatrix{
   \coprod_{\alpha\in I_n}S^{n-1} \ar@{^(->}[r] \ar[d]_{\coprod \phi_\alpha}
   \ar@{}[dr]|(.75)\po
   & \coprod D^n \ar[d]^{\Phi_\alpha}\\
   X^{(n-1)}\ar[r] & X^{(n)} }
 \]
 for some indexing set $I_n$. The $\phi_\alpha$ are called \emph{attaching maps} and the
 $\Phi_\alpha$ are called \emph{characteristic maps}. Define $e_\alpha^n =
 \Phi_\alpha(\inn D^n)$. Define $X$ as $\bigcup X^{(n)}$, with the direct limit topology
 (a set is open if and only if the intersection with $X^{(n)}$ is open for each $n$).

 \section{$\pi_k$ and more CW stuff}
 \begin{definition}
   $\pi_k(X,x_0) := \bigl[ (S^k,s_0),(X,x_0)\bigr] = \pi_0\bigl( (X,x_0)^{(S^k,x_0)}\bigr)$.
 \end{definition}
 Note that $S^k\approx D^k/\partial D^k \approx I^k/\partial I^k$. For $k\ge 1$,
 $\pi_k(X,x_0)$ has a group structure given by
 \begin{gather*}
    f\cdot g: I^k = I^{k-1}\times I \to I^{k-1} \times 2I=I^k \cup I^k
    \xrightarrow{f\cup g} X\\
    \begin{xy} 0 *=(1,1)\frm{-} *{f} \end{xy} \cdot
    \begin{xy} 0 *=(1,1)\frm{-} *{g} \end{xy} =
    \begin{xy}
      0 *=(1,.5)!UR\frm{-} *=(1,.5)!DR\frm{-} *+!L{I};
      (-.5,.25) *{f}; (-.5,-.25) *{g}; (-.5,-.5) *+!U{I^{k-1}}
    \end{xy}
 \end{gather*}
 \begin{lemma} For a CW complex $X$,
   \begin{enumerate}
     \item $\Phi_\alpha(D^n)=\bar e^n_\alpha$ (closure in $X$).
     \item For $A\subseteq X$, $\Phi_\alpha^{-1}(A)$ is closed if and only if $A\cap \bar
     e^n_\alpha$ is closed.
     \item $X$ has the weak topology with respect to the maps $\Phi_\alpha:D^n\to X$.
   \end{enumerate}
 \end{lemma}
 \begin{proof}
   (1) $\Phi_\alpha(D^n)$ is closed (since $D^n$ is compact and $X$ is Hausdorff (by
   homework 3)) and contains $e^n_\alpha$. If $A$ is closed, with $e^n_\alpha\subseteq
   A\subseteq \Phi_\alpha(D^n)$, then $\Phi_\alpha^{-1}(A)$ is a closed set in $D^n$
   which contains $\inn D^n$, so it is all of $D^n$. It follows that
   $A=\Phi_\alpha(D^n)$.

   (2) Is as easy as (1).

   (3) If $A\subseteq X$ has the property that $\Phi_\alpha^{-1}(A)\subseteq D^n$ is
   closed for all $\alpha$, then we'd like to show that $A\cap X^{(n)}$ is closed for all
   $n$. Well, $A\cap X^{(-1)}=\varnothing$ is closed. Now induct; assume $A\cap
   X^{(n-1)}$ is closed. Then using the property of push-out, $A\cap X^{(n)}$ is closed.
 \end{proof}

 \section{Van Kampen's Theorem}
 Chris lectures today because Peter is ill. Remember the group law on $\pi_1(X)$ and
 that $\pi_1$ is a functor, sending homotopy equivalences to isomorphisms. Say we have a
 cover $j_\alpha:A_\alpha\to X$ of $X$ by open sets $A_\alpha$, all of which contain the
 base point $x_0$. Then it is clear that we have commutative squares
 \[
 \xymatrix{
  A_\alpha\cap A_\beta \ar@{^(->}[r]^(.65){i_{\alpha\beta}} \ar@{^(->}[d]_{i_{\beta\alpha}} &
  A_\alpha \ar@{^(->}[d]^{j_\alpha}\\
  A_\beta \ar@{^(->}[r]^{j_\beta} & X
 }\qquad\raisebox{-1pc}{\text{ which induce }}\qquad \xymatrix{
  \pi_1(A_\alpha\cap A_\beta) \ar[r]^(.55){i_{\alpha\beta *}}
  \ar[d]_{i_{\beta\alpha *}} & \pi_1(A_\alpha) \ar[d]^{j_{\alpha *}}\\
  \pi_1(A_\beta) \ar[r]^{j_{\beta *}} & \pi_1(X)
 }\]
 where each space has the base point $x_0$. Thus, we have a map $\Phi:\bigast_\alpha
 \pi_1(A_\alpha,x_0) \to \pi_1(X)$, and it is obvious the relations
 $j_{\alpha *}\circ i_{\alpha \beta *} (w) = j_{\beta *} \circ i_{\beta \alpha *} (w)$.

 \begin{theorem}[Van Kampen's Theorem]
   Let $X=\bigcup A_\alpha$, with $X$, $A_\alpha$, and $A_\alpha\cap A_\beta$ all path
   connected and containing the base point $x_0$. Then the map $\Phi:\bigast_\alpha
   \pi_1(A_\alpha,x_0)\to \pi_1(X,x_0)$ is surjective. Furthermore, if all of the
   $A_\alpha\cap A_\beta\cap A_\gamma$ are path connected, then the kernel of $\Phi$ is
   generated by elements of the form $i_{\alpha \beta *}(w)\cdot i_{\beta\alpha
   *}(w)^{-1}$, where $w\in \pi_1(A_\alpha\cap A_\beta,x_0)$.
 \end{theorem}
 \begin{proof}
   First we prove surjectivity. Let $f:I\to X$ be a loop at $x_0$. Choose $0=s_0<
   s_1<\cdots < s_{n-1}< s_n=1$ so that $f([s_i,s_{i+1}])\subseteq A_{\alpha_i}=:A_i$.
   Define $f_i$ to be the path $I\xrightarrow{\sim} [s_i,s_{i+1}]
   \xrightarrow{f|_{[s_i,s_{i+1}]}} X$. Choose paths $g_i:I\to A_i\cap A_{i+1}$ from
   $f(s_i)$ to $x_0$. Then $f\simeq (f_0 g_1)(g_1^{-1}f_1g_2)\cdots (g_{n-2} f_{n-2}
   g_{n-1}) (g_{n-1}^{-1} f_{n-1})\in \im \Phi$.

   To prove that the kernel of $\Phi$ is what we want, it is enough to show that given an
   element $f_1\cdots f_n\in \bigast_\alpha \pi_1(A_\alpha)$, with $f_i\in A_{\alpha_i}$,
   such that $\Phi(f_1\cdots f_n)=0$, we can turn $f_1\cdots f_n$ into the constant map
   using a series of the following moves:
   \begin{enumerate}
     \item Replace $f_i$ by an equivalent element of $\pi_1(A_{\alpha_1})$ (i.e. homotope
     $f_i$ within $A_{\alpha_i}$), or, if $\alpha_i=\alpha_{i+1}$, replace $f_i\cdot
     f_{i+1}$ by their product in $\pi_1(A_{\alpha_i})$. These operations don't change
     the element of $\bigast_\alpha \pi_1(A_\alpha)$.

     \item If the image of $f_i$ lies in $A_\beta$, then think of it as an element of
     $\pi_1(A_\beta)$, rather than an element of $\pi_1(A_{\alpha_i})$.
   \end{enumerate}
   Let $F:I\times I\to X$ be a homotopy from $f_1\cdots f_n$ to the constant map. Then
   cut up $I\times I$ into little squares, so that the image of each square lies entirely
   within some $A_\alpha$. We can perturb the squares slightly so that each point touches
   at most three of the squares:
   \[\begin{xy}
     (-1.5,-1) *{1} *=(1,1)\frm{-},
     (-.5,-1) *{2} *=(1,1)\frm{-},
     (.5,-1) *{3} *=(1,1)\frm{-},
     (1.5,-1) *{4} *=(1,1)\frm{-},
     (-1.5,0) *{5} *=(1.1,1)!(-.05,0)\frm{-},
     (-.4,0) *{6} *=(1,1)\frm{-},
     (.6,0) *{7} *=(1,1)\frm{-},
     (1.6,0) *{8} *=(.9,1)!(.05,0)\frm{-},
     (-1.5,1) *{9} *=(1,1)\frm{-},
     (-.5,1) *{10} *=(1,1)\frm{-},
     (.5,1) *{11} *=(1,1)\frm{-},
     (1.5,1) *{12} *=(1,1)\frm{-},
   \end{xy}\]
   For each vertex $v$, choose a path $g_v$ from $F(v)$ to $x_0$. We may choose the path
   $g_v$ to lie entirely in the three open sets containing the images of the squares
   adjacent to $v$. Number the little squares as shown, and let $\gamma_i$ be the path
   which has little squares $1$ through $i$ below it, so $\gamma_0$ is the bottom edge,
   with $F(\gamma_0)=f_1\cdots f_n$. Notice that we get a factorization of $\gamma_i$ as
   an element of $\bigast_\alpha \pi_1(A_\alpha)$ by looking at the images of the
   horizontal and vertical edges and concatenating with the $g_v$. But it is better than
   that: the image of each edge lies in two of the $A_\alpha$! That is, we get two
   versions of each edge, related by operation 2 above.

   Start with $\gamma_0$, written as some element of $\bigast \pi_1(A_\alpha)$. Apply a
   homotopy to the first factor that is entirely in $A_1$ (the open set containing the
   image of the square labeled 1). Now think of the new edges as living in the adjacent
   $A_\alpha$ (the is move 2), and repeat until you've reached the constant map.
 \end{proof}
 \begin{example}
   Let $n>1$. In $S^n$, let $U$ be a neighborhood of the closed northern hemisphere
   ($U\approx \inn D^n$), let $V$ be a neighborhood of the closed southern hemisphere
   ($V\approx \inn D^n$), and let the base point be on the equator. Then $U\cap V$ is
   path connected (not true if $n=1$), so by Van Kampen's Theorem, $\pi_1(\inn D^n)\ast
   \pi_1(\inn D^n)\cong \{e\}$ surjects onto $\pi_1(S^n)$, so $\pi_1(S^n)$ is trivial.
 \end{example}
 \begin{example}
   If $X=\bigvee_\alpha X_\alpha$, where each $X_\alpha$ is path connected and for each
   $\alpha$, $x_\alpha\in X_\alpha$ is a deformation retract of a neighborhood. Then Van
   Kampen's Theorem applies to tell us that $\pi_1(X)\cong \bigast_\alpha
   \pi_1(X_\alpha)$.
 \end{example}
 \begin{example}
   Write the torus as the union of a popped torus and a patch:
   \[\begin{pspicture}(-.5,-.5)(.5,.5)
     \pspolygon[fillstyle=solid,fillcolor=lightgray](-.5,-.5)(.5,-.5)(.5,.5)(-.5,.5)
   \end{pspicture} \raisebox{1pc}{\qquad =\qquad}
   \begin{pspicture}(-.5,-.5)(.5,.5)
     \pspolygon[fillstyle=solid,fillcolor=lightgray](-.5,-.5)(.5,-.5)(.5,.5)(-.5,.5)
     \pscircle[fillstyle=solid,fillcolor=white,linestyle=dashed](0,0){.2}
   \end{pspicture}\raisebox{1pc}{\qquad $\cup$ \qquad}
   \begin{pspicture}(-.5,-.5)(.5,.5)
     \pscircle[fillstyle=solid,fillcolor=lightgray,linestyle=dashed](0,0){.4}
   \end{pspicture}\]
  The first piece, call it $U_1$, is homotopic to the wedge of two circle, so
  $\pi_1(U_1)=\ZZ\ast \ZZ$, generated by $a$ and $b$. The second, call it $U_2$, is
  contractible, so $\pi_1(U_2)=1$. The intersection is homotopic to a circle,
  $\pi_1(U_1\cap U_2)=\ZZ$. The generator of this $\ZZ$ is sent to $1$ is $\pi_1(U_2)$
  (of course), and to $aba^{-1}b^{-1}$ in $\pi_1(U_1)$. Thus, $\pi_1(\mathbb{T}^2) =
  \ZZ\ast\ZZ/(aba^{-1}b^{-1}=1) \cong \ZZ\oplus \ZZ$.
 \end{example}
 We will see later that given any finitely presented group $G=\langle g_1,\dots,
 g_n|r_1,\dots, r_m \rangle$, one can come up with a topological space (in fact, a CW
 2-complex) with the desired group as a fundamental group.

 \section{Characterization of CW complexes}
 We mention the following lemma.
 \begin{lemma}
   If a finite group $G$ acts freely on a manifold $M^n$, then $M^n/G$ is a manifold of
   dimension $n$.
 \end{lemma}
 \begin{lemma}\label{L:cmptinCW}
   If $X$ is a CW complex and $C\subseteq X$ is compact, then $C$ lies in a finite
   subcomplex.
 \end{lemma}
 \begin{proof}
   For each cell intersecting $C$, choose a point in the intersection. Let $S$ be the
   union of these points. Since the points are in distinct cells, $S$ is discrete, so
   closed. Since $S\subseteq C$ must be compact, $S$ must be finite.

   Now it remains to show that the closure of each cell lies in a finite subcomplex. The
   image of the attaching map of an $n$-cell is a compact and contained in $X^{(n-1)}$,
   so by induction on the dimension, it lies in a finite subcomplex. Clearly, any 0-cell
   \emph{is} a finite subcomplex.
 \end{proof}
 \begin{theorem}[Characterization of CW complexes]
   $X$ is a CW complex if and only if
   \begin{enumerate}
     \item[0.] $X$ is Hausdorff.

     \item
     T\smash{\raisebox{-2.5ex}{\llap{``C'' $\left\{ \rule{0pt}{4ex} \right.$\qquad}}}here
     are $\phi_\alpha:D^n\to X$ such that $\phi_\alpha:\inn D^n\xrightarrow{\approx}
     \phi_\alpha(\inn D^n) =: e^n_\alpha$, with $X= \coprod_{n,\alpha} e^n_\alpha$.

     \item $\phi_\alpha(\partial D^n)$ lies in a finite union of lower-dimensional cells.
     (``closure finite'')

     \item $X$\llap{``W'' $\{$ \qquad} has the weak topology with respect to the $\phi_\alpha$.
   \end{enumerate}
 \end{theorem}
 \begin{proof}
   ($\Rightarrow$) We've checked 0, 1, and 3 already. 2 follows from Lemma
   \ref{L:cmptinCW}.

   ($\Leftarrow$) In Hatcher.
 \end{proof}
 \begin{definition}
   $A\subseteq X$ is a \emph{retract} if there is a \emph{retraction} $r:X\to A$, a map
   so that $r|_A=\id_A$.
 \end{definition}
 \begin{definition}
   $A\subseteq X$ is a \emph{deformation retraction} if there is a \emph{deformation
   retraction} $r:X\to A$, a retraction such that $i_{A\hookrightarrow X}\circ r\simeq
   \id_X$. In particular, $A\simeq X$.
 \end{definition}
 \begin{definition}
   $A\subseteq X$ is a \emph{strong deformation retraction} if there is a \emph{strong
   deformation retraction} $r:X\to A$, a deformation retraction so that $i\circ r\simeq_A
   \id_X$. In particular $X\simeq_A A$.
 \end{definition}
 \begin{example}
   A semi-circle in a circle is a retract, but not a deformation retract.
 \end{example}

 \section{The Homotopy Lemma}
 \begin{lemma}
   $S^{n-1}\times I \cup D^n\times \{0\}$ is a strong deformation retract of $D^n\times
   I$.
 \end{lemma}
 \begin{proof}
   \[\psset{fillstyle=solid,fillcolor=lightgray}
   \begin{pspicture}(0,0)(3,1)
     \psline[fillstyle=none](0,1)(0,0)(1,0)(1,1)
     \rput(1.5,.5){$\subseteq$}
     \psline(2,1)(2,0)(3,0)(3,1)
   \end{pspicture}\qquad \qquad \qquad
   \begin{pspicture}(0,0)(1,1.5)
     \rput(.5,1.2){$\bullet$}
     \psline(0,1)(0,0)(1,0)(1,1)
     \psline(.5,1.2)(0,.2)
     \rput(0,.2){$\bullet$} \uput[180](0,.2){$\phi(x)$}
     \rput(.2,.6){$\bullet$} \uput[-10](.2,.6){$x$}
   \end{pspicture}\qedhere\]
 \end{proof}
 More generally,
 \begin{lemma}\label{L:CWhep}
   If $(Z,B)$ is a CW pair, then $B\times I\cup Z\times \{0\}$ is a strong deformation
   retract of $Z\times I$.
 \end{lemma}
 \begin{proof}
   Apply the previous lemma repeatedly. Induct on dimension. \anton{how about if $Z$ is
   not finite dimensional}
 \end{proof}
 \begin{lemma}[Homotopy Lemma]
   If $(Z,B)$ is a CW pair, and if $f_0,f_1:B\to Y$ are homotopic, then $Y\cup_{f_0} Z$
   is homotopic to $Y\cup_{f_1} Z$.
 \end{lemma}
 \begin{proof}
   Let $F:B\times I\to Y$ be a homotopy $f_0\simeq f_1$. We wish to show that
   $Y\cup_{f_1} Z\subseteq Y\cup_F (Z\times I)$ is a strong deformation retract for
   $i=0,1$. By the previous lemma, $B\times I\cup Z\times \{i\}$ is a strong deformation
   retract of $Z\times I$. This induces a strong deformation retract of $Y\cup_F (Z\times
   I)$:
   \[\begin{pspicture}(-3,.5)(4,2.5)
     \pscurve(1,1)(1.3,1.3)(1.7,1)(2,1.1)
     \pscurve(2,2.1)(1.7,2)(1.3,2.3)(1,2)
     \psline[linestyle=dotted](1,1)(1.2,1.4)(1.2,2.4)
     \psline(1,1)(1,2)(1.2,2.4)
     \psline(2,2.1)(2,1.1)(2.2,1.5)(2.2,2.5)(2,2.1)
     \pscurve[linestyle=dotted](1.2,1.4)(1.5,1.7)(1.9,1.4)(2.2,1.5)
     \pscurve(1.2,2.4)(1.5,2.7)(1.9,2.4)(2.2,2.5)
     \psline(.95,1.8)(.5,1.8)(0,.5)(2.5,.5)(3,1.8)(2.25,1.8)
     \uput[180](0,1.5){$Y\cup_F (Z\times I) = $}
     \rput(3.6,1.5){$\longrightarrow$}
     \uput{.5}[30](0,.5){$Y$}
   \end{pspicture}
   \begin{pspicture}(0,.5)(4,2.5)
     \pscurve(1,1)(1.3,1.3)(1.7,1)(2,1.1)
%     \pscurve(2,2.1)(1.7,2)(1.3,2.3)(1,2)
     \psline(1,1)(1.2,1.4)(1.2,2.4)
     \psline(1,1)(1,2)(1.2,2.4)
     \psline(2,1.1)(2.2,1.5)
     \pscurve(1.2,1.4)(1.5,1.7)(1.9,1.4)(2.2,1.5)
%     \pscurve(1.2,2.4)(1.5,2.7)(1.9,2.4)(2.2,2.5)
     \psline(.95,1.8)(.5,1.8)(0,.5)(2.5,.5)(3,1.8)(1.25,1.8)
     \uput[0](3,1.5){$= Y\cup_{f_i} Z$}
     \uput{.5}[30](0,.5){$Y$}
     \psline[border=1pt]{->}(3,.5)(1.6,1.3)
     \uput[0](3,.5){$F(B\times I)$}
   \end{pspicture}\qedhere\]
 \end{proof}
 In particular, if $B$ is a space, and $\alpha\in \pi_{n-1}(B,b_0)$, then $B\cup_\alpha
 D^n$ is well defined up to homotopy equivalence.
 \begin{definition}
   A pair $(X,A)$ has the \emph{homotopy extension property} (\emph{HEP}) if a homotopy
   of $A$, defined at time zero on $X$, extends to all of $X$:
   \[\xymatrix{
    (X\times \{0\})\cup_{A\times \{0\}} (A\times I) \ar[r]^(.8)h &Y\\
    X\times I\ar@{<-^)}[u] \ar@{.>}[ur]_{\tilde h}
   }\]
 \end{definition}
 For example, any CW pair has HEP by Lemma \ref{L:CWhep}.
 \begin{lemma}
   $(X,A)$ has HEP if and only if $X\times \{0\} \cup A\times I$ is a retract of $X\times
   I$.
 \end{lemma}
 \begin{proof}
   ($\Rightarrow$) Let $Y=X\times \{0\}\cup A\times I$ with $h=\id$. Then $\tilde h$ is a
   retract.

   ($\Leftarrow$) If $r$ is a retract, set $\tilde h = h\circ r$.
 \end{proof}
 \begin{lemma}
   If $(X,A)$ has HEP, then $A\subseteq X$ is closed.
 \end{lemma}
 Some applications:
 \begin{enumerate}
  \item If $(X,A)$ has HEP and the inclusion $A\hookrightarrow X$ is a homotopy
  equivalence, then $A$ is a deformation retract of $X$.

  \item If $(X,A)$ has HEP and $A$ is contractible, then the canonical map $X\to X/A$ is
  a homotopy equivalence. To see this, let $h:A\times I\to A$ be a contraction of $A$, so
  $h_0=\id_A$ and $h_1(A)$ is a point, then extend $h_0$ to $\id_X$. By HEP, there is a
  homotopy $\tilde h:X\times I\to X$ so that $h_0=\id_X$ and $h_1(A)$ is a point. Then
  $h_1$ induces a map $\phi:X/A\to X$, and $h$ induces a homotopy between $\id_{X/A}$ and
  the composition $X/A\xrightarrow{\phi}X\to X/A$. Also, $h$ is itself a homotopy between
  $\id_X$ and the composition $X\to X/A\xrightarrow{\phi} X$.
 \end{enumerate}
 \begin{example}
   In a connected CW complex $X$, one can find a maximal spanning tree $T\subseteq X$, a
   contractible subcomplex of $X^{1}$. Then application 2 above says that $T$ can be
   crushed to a point without changing the homotopy type of $X$. Moreover, it is easy to
   check that $X/T$ is a CW complex with a single $0$-cell.

   Later we will see that for a CW complex $X$, if $\pi_k(X)=0$, $k>1$, then $X$ is
   homotopic to a CW complex with no $k$-cells. \anton{is this right?}
 \end{example}

 \section{Some Theorems}
 Chris lectures.

 We will assume we know
 \begin{itemize}
   \item $\pi_i(D^n)=0$ for all $i$ since $D^n$ is contractible.
   \item $\pi_1(S^n)=0$ for $n>1$.
   \item $\pi_1(S^1)=\ZZ$, generated by $\id:S^1\to S^1$.
   \item $\pi_i(S^n)=0$ for $i< n$.
   \item $\pi_n(S^n)=\ZZ$, generated by $\id:S^n\to S^n$.
   \item If $f:S^n\to S^n$ satisfies $f(-x)=-f(x)$, then $[f]\neq 0\in \pi_n(S^n)$; in
   fact, $[f]$ must represent an odd integer.
 \end{itemize}
 Then we can prove some neat things.
 \begin{theorem}[Fundamental Theorem of Algebra]
   Every non-constant $p\in \CC[x]$ has a root in $\CC$.
 \end{theorem}
 \begin{proof}
   Assume not. Say $p$ is of degree $n$. Note that $p_R=p\bigl( \{z| \|z\|=R\} \bigr)$ is
   a loop in $\CC\smallsetminus \{0\}\simeq S^1$, so it defines an element of
   $\pi_1(\ZZ)$.\footnote{Since we didn't choose a base point, it only defines a
   conjugacy class, but $\ZZ$ is abelian, so we get an element.} For $R$ large, it is
   easy to see that it defines the same element as does $z^n$, which is $n\in \ZZ$.
   Slowly shrinking $R$ to zero, we get a homotopy of $p_R$ to $p_0$, which is a constant
   map, corresponding to $0\in \ZZ$. Thus, $n=0$, contradicting the assumption that $p$
   is non-constant.
 \end{proof}
 \begin{theorem}[Brauwer Fixed Point Theorem]
   Every $h:D^n\to D^n$ has a fixed point.
 \end{theorem}
 \begin{proof}
   Assume not, then define $f:D^n\to S^{n-1}$ by the picture
   \[
   \begin{xy}
     0 *\xycircle(1,1){},
     (.2,.3) *{\bullet} *+!UL{x},
     (-.5,-.3) *{\bullet} *+!UL{h(x)};
     a(47) *{\bullet} *+!DL{f(x)} **@{-}
   \end{xy}
   \]
   It is easy to see that $f$ is continuous and that $f|_{S^n}=\id_{S^n}$. Then we get
   the following commutative triangle, and its image under $\pi_{n-1}$.
   \[\xymatrix{
    S^{n-1} \ar@{^(->}[r] \ar@/_2ex/[rr]_{\id_{S^{n-1}}} & D^n \ar[r]^f & S^{n-1}
   }\qquad\qquad
   \xymatrix{
    \ZZ \ar[r] \ar@/_2ex/[rr]_{\id_\ZZ} & 0 \ar[r]^{f_*} & \ZZ
   }\]
   Which is impossible.
 \end{proof}
 \begin{theorem}[Invariance of Dimension]
   $\RR^n\approx \RR^m \Longleftrightarrow n=m$.
 \end{theorem}
 \begin{proof}
   ($\Leftarrow$) is obvious. Let's prove ($\Rightarrow$). Assume
   $f:\RR^n\xrightarrow{\sim} \RR^m$, then we get a homeomorphism $f:\RR^n\smallsetminus
   \{0\} \xrightarrow{\sim} \RR^m\smallsetminus\{f(0)\}$. The first is homotopy
   equivalent to $S^n$ and the second to $S^m$, so it follows that $S^n\simeq S^m$. We
   may assume $n\le m$. Then $\ZZ=\pi_n(S^n)\cong \pi_n(S^m)$, which is zero if $n<m$. So
   $n=m$.
 \end{proof}
 \begin{theorem}[Borsuk-Ulam]
   For any $f:S^n\to \RR^n$, there is some $x\in S^n$ so that $f(x)=-f(-x)$.
 \end{theorem}
 \begin{proof}
   If not, define $g:S^n\to S^{n-1}$ by $g(x) = \frac{f(x)-f(-x)}{\|f(x)-f(-x)\|}$. Note
   that $g(-x)=-g(x)$, so composing with the inclusion $i:S^{n-1}\hookrightarrow S^n$, we
   get a non-trivial element of $\pi_n(S^n)=\ZZ$. But if $[i\circ g]\neq 0$, then the
   image $i\circ g(S^n)$ is not contractible in $S^n$, so the image $i(S^{n-1})$ is not
   contractible in $S^n$. Contradiction.
 \end{proof}
 Note that we don't really use much about what $\pi_1$ \emph{is} in these proofs, we just
 use some functorial properties. So anything else that behaves kind of like $\pi_1$ would
 work just as well. Later we will define homology and cohomology groups, which serve
 exactly this purpose.

 \section{Boring class}
 Peter was sick, but still in class, and asked people to volunteer information about
 \begin{itemize}
   \item categories and functors
   \item exact squences
   \item cofibrations and Puppe sequences
   \item covering spaces, fiber bundles, and fibrations
 \end{itemize}
 We only talked about the first two.

 \section{More Category stuff, Cofibrations, Fibrations}
 \begin{definition}
   An sequence of pointed sets
   \[
    (A,a)\xrightarrow{f} (B,b)\xrightarrow{g} (C,c)
   \]
   is said to be \emph{exact at $B$} if $\im f = g^{-1}(c)$.
 \end{definition}
 Note that if we take these to be groups (with the identity element), we get the usual
 notion of exactness of a sequence of groups.
 \begin{definition}
   A \emph{natural transformation} $\eta:F\to G$ between two functors $F,G:\mathcal{C}\to
   \mathcal{D}$ is a morphism $\eta(X)$ for each object $X$ in $C$ so that for every
   $f:X\to Y$, we have
   \[\xymatrix{
    F(X) \ar[r]^{\eta(X)} \ar[d]_{Ff} \ar@{}[dr]|{\circlearrowleft} & G(X)\ar[d]^{Gf}\\
    F(Y) \ar[r]^{\eta(Y)} & G(Y)
   }\]
 \end{definition}
 \begin{example}
   For any object $X$ in a category $\C$, we get a functor $h^X:\C^\text{op}\to
   \textbf{Set}$ given by $h^X(Y)=\hom(Y,X)$. Similarly, we get a functor $h_X:\C \to
   \textbf{Set}$ given by $h_X(Y)=\hom(X,Y)$. We call a functor \emph{representable} if
   it is isomorphic to $h^X$ or $h_X$ for some $X$.
 \end{example}
 \begin{theorem}[Yoneda's Lemma]
   For any functor $F:\C^\text{op}\to \textrm{\bf Set}$, there is a natural
   bijection ${\rm Nat}(h^X,F)\cong F(X)$. In particular, taking $F=h^Y$, we see that
   the functor $h^{-}:\C\to {\rm Fun}(\C^\text{op},\textrm{\bf Set})$ is a fully
   faithful embedding of categories.
 \end{theorem}
 Similarly, we get a fully faithful embedding $h_{-}:\C^\text{op}\to
 \textrm{Fun}(\C,\textbf{Set})$. This is the Yoneda embedding of $\C^\text{op}$.
 \begin{proof}
   Given $\eta\in \textrm{Nat}(h^X,F)$, we have $\eta(X):\hom(X,X)\to F(X)$, so we get an
   element $a=\eta(X)(\id_X)\in F(X)$. Conversely, given $a\in F(X)$, we construct a
   natural transformation $\eta$ which takes $f\in h^X(Y)=\hom(Y,X)$ to
   $\eta(Y)(f)=(Ff)(a)$. Check that these are inverses, and that the bijection is natural
   in $F$ and $X$. The following diagram should help:
   \[\raisebox{6.5pc}{$\xymatrix @C=13pc @R=5pc{
   \id_X \ar@{|->}[d] \ar@{}[dr]|(.1){
      \xymatrix @!0 @R=3pc @C=6.5pc{
      \hom(X,X)\ar[d]_{- \circ f} & F(X) \ar[d]^{Ff}\\
      \hom(Y,X) \ar[r]^(.6){\eta(Y)} & F(Y)}
   } & a\ar@{|->}[d]\\
   f \ar@{|->}[r] & (Ff)(a)
   }$}\qedhere\]
 \end{proof}
 Note that representable functors sometimes factor through some other category on their
 way to \textbf{Set}, i.e.\ there is a natural group structure on $h^X(Y)$ in
 such a way that the maps $h^X(Z)\to h^X(Y)$ induced by $Y\to X$ are group homomorphisms.
 \begin{definition}
   If $\C$ has products, then an object $K$ is a \emph{group object} in $\C$ if there is
   a maps $m:K\times K\to K$ such that there exist $i:K\to K$ and $1:*\to K$, where $*$
   is the final object of $\C$ (the empty product) satisfying the following diagrams.
   \[\xymatrix{
    {*}\times K\ar[d]_{1\times \id} \ar@{}[r]|(.6){\cong} & K\ar@{}[r]|(.4){\cong} \ar[d]^{\id}
    & K\times *\ar[d]^{\id\times 1}\\
    K\times K\ar[r]^(.6)m & K & K\times K \ar[l]_(.6)m
   }\quad
   \xymatrix{
    K\times K \ar[d]_m &
    K \ar[d]^{\exists !} \ar[r]^(.4){\id\times i} \ar[l]_(.35){i\times \id}
    & K\times K \ar[d]^m\\
    K & {*}\ar[r]^1 \ar[l]_1 & K
   }\quad
   \xymatrix{
    K\times K\times K \ar[r]^(.57){m\times \id} \ar[d]_{\id \times m} & K\times K\ar[d]^m\\
    K\times K\ar[r]^(.6)m & K
   }\]
   If we change all the products to coproducts (in particular, change the final object to
   the initial object) and reverse the arrows, we have the definition of a \emph{cogroup
   object}.
 \end{definition}
 Observe that once $i$ and $1$ exist, they are unique. Note that if $K$ is a group
 (resp.\ cogroup) object, then $h_K$ (resp. $h^K$) factors trough
 \textbf{Gp}.\footnote{In fact, $h^K$ factors through \textbf{Gp} exactly when it is a
 group object in $\textrm{Fun}(\C,\textbf{Set})$, so by Yoneda's Lemma, $h^K$ factors
 through \textbf{Gp} if and only if $K$ is a group object. Similarly, $h_K$ factors
 through \textbf{Gp} if and only if $K$ is a cogroup object.} If the multiplication map
 $m$ is invariant under the ``switch factors'' map, then we says that $K$ is an
 \emph{abelian} (co)group object. In this case, the representable functors factor through
 \textbf{Ab}.
 \begin{definition}
   An H-group is a group object in \textbf{hTop}. Note that the identity element,
   inverses, and associativity of the product only work up to homotopy!
 \end{definition}
 \begin{example}
   $(S^n,*)$ is an H-cogroup for $n\ge 1$, with comultiplication given by the ``crush the
   equator'' map $S^n\to S^n\vee S^n$ and inverse given by ``reflect through the
   equator'' map $S^n\to S^n$. If $n\ge 2$, then it is an abelian H-cogroup.

   The representable functor $h_{(S^n,*)}$ is $\pi_n:\textbf{hTop}_{pt}\to \C$ is , where
   $\C=\textbf{Set}$, \textbf{Gp}, or \textbf{Ab} when $n=0$, 1, or $n\ge 2$,
   respectively.
 \end{example}
 \begin{example}
   Later we'll se that for an abelian group $A$, there are abelian H-groups $K(A,n)$,
   called \emph{Eilenberg-Maclane} spaces, defined by
   $\pi_m\bigl(K(A,n)\bigr)=\delta_{m,n} A$. We will define \emph{cohomology functors}
   $H^n(X,A)$ which will turn out to be equal to $h^{K(A,n)}:\textbf{hTop}_{pt}\to
   \textbf{Ab}$.
 \end{example}
 \begin{definition}
   A map $i:A\to X$ is a \emph{cofibration} if it has the homotopy extension property,
   i.e.\ the diagram on the left for all $Y$, $h$, and $\tilde h_0$. This means that
   any homotopy of maps from $A$ can be extended to a homotopy of maps from $X$, given a
   starting point.
   \[\xymatrix{
    A\ar[d]_{i} \ar[r]^h & Y^I \ar[d]^{0\text{ endpoint}}\\
    X \ar[r]_{\tilde h_0} \ar@{-->}[ru]^{\exists \tilde h} & Y
   }\qquad\qquad
   \xymatrix{
    Y\ar[r]^{\tilde h_0} \ar[d]_{\id\times \{0\}} & E\ar[d]^p\\
    Y\times I \ar[r]_{h} \ar@{-->}[ru]^{\exists \tilde h}& B }\] A map $p:E\to B$ is a
   \emph{fibration} if it satisfies the \emph{homotopy lifting property} (HLP), i.e.\ the
   diagram on the right for all $Y$, $h$, and $\tilde h_0$. This means that any homotopy
   of maps to $B$ can be lifted to a homotopy of maps to $E$, given a starting point. We
   say that $p:E\to B$ is a \emph{Serre fibration} if it satisfies HLP when $Y=D^n$ (or,
   equivalently, when $Y$ is a CW-complex).
 \end{definition}
 If $B$ is a pointed space with base point $b$, and $E\xrightarrow{p} B$ is a fibration,
 then we define the fiber $F:=p^{-1}(b)$. It turns out that a cofibration is always an
 inclusion of a closed subset.

 \begin{lemma} \label{Lcofibrationexact}
   If $i:A\to X$ is a cofibration, then for each pointed space $(Y,y)$ we get an
   exact sequence of sets
   \[[X/A,Y]\to [X,Y]\to [A,Y].\]
 \end{lemma}
 \begin{proof}
   It is clear that the composition of maps always lands in the base point of $[A,Y]$. If
   $f:X\to Y$ is a map such that $f|_A$ homotopic to the constant map $A\to y$. Then we
   have a homotopy of maps $f|_A\simeq \text{const}_y$ from $A$. Since $i$ is a
   cofibration, we can extend to a homotopy $f\simeq g$, where $g:X\to Y$ and
   $g|_A=\text{const}_y$, so $g$ induces a map $X/A\to Y$. This proves exactness.
 \end{proof}
 \begin{lemma} \label{L:fibrationexact}
   If $p:E\to B$ is a fibration with fiber $F$, then for each $Y$, we get an exact
   sequence of sets
   \[[Y,F]\to [Y,E]\to [Y,B].\]
 \end{lemma}
 \begin{proof}
   It is clear that the composition of maps always lands in the base point of $[Y,B]$. If
   $f:Y\to E$ is a map such that $p\circ f\simeq \text{const}_b$, then we can lift to a
   homotopy $f\simeq g:Y\to E$, where $p\circ g=\text{const}_b$. That is, the image of
   $g$ is in $F$. This proves exactness.
 \end{proof}

 \begin{lemma}
   A pushout of a cofibration is a cofibration. A pull back of a (Serre) fibration is a
   (Serre) fibration.
 \end{lemma}
 \begin{proof}
   The curved dashed arrows exist because $A\to X$ is a cofibration and $E\to B$ is a
   (Serre) fibration (with $Y$ CW, in the Serre fibration case).
   \[\xymatrix{
    A \ar[r] \ar[d] \ar@{}[dr]|(.75)\po & A'\ar[d] \ar[r] &  Y^I\ar[d] \\
    X\ar[r] \ar@/^/@{-->}[rru]|!{[ru];[r]}\hole & X'\ar[r] \ar@{-->}[ru] & Y
    %\ar@{.>} `d[r] `{[rrru]-(.57,0)} `[rru] [rru]
   }\qquad
   \xymatrix{
    Y\ar[d] \ar[r] & E'\ar[d] \ar@{}[dr]|(.25)\pb \ar[r] & E\ar[d]\\
    Y\times I\ar[r] \ar@/_/@{-->}[rru]|!{[ru];[r]}\hole \ar@{-->}[ru] & B'\ar[r] & B
   }\]
   The straight dashed arrows exist by the universal properties of pull-back and
   push-out.
 \end{proof}

% \begin{theorem}\label{T:fibrationLEShomotopy}
%   If $F\to E\xrightarrow{p} B$ is a Serre fibration, then there is a long exact sequence
%   \[
%    \pi_{n+1}(B,b)\xrightarrow{\ \delta\ }
%    \pi_n(F,e)\xrightarrow{i_*} \pi_n(E,e)\xrightarrow{p_*} \pi_n(B,b)\xrightarrow{\ \delta\ }
%     \pi_{n-1}(F,e)\to \cdots
%    \to \pi_0(B,b)
%   \]
%%   \[\xymatrix{
%%    & \pi_n(F,e) & \pi_n(E,e) & \pi_n(B,b)
%%    \ar `r[d]
%%        `[ll]
%%        `l[dlll]
%%        `d[dll]
%%        [dll]\\
%%   & \pi_{n-1}(F,e) & &
%%   }\]
% \end{theorem}
% \begin{proof}
%   Lemma \ref{L:fibrationexact} proves exactness at $\pi_n(E,e)$ for all $n$ (taking $Y$
%   to be $S^n$).
%
%   What is the map $\delta$? If we have $f:(I^n,\partial I^n)\to (B,b)$,
%   then we can lift a face of $I^n$ to the point $e\in E$. Since $p:E\to B$ is a
%   fibration, we get a lift $\tilde f:I^n\to E$ which sends $\partial I^n\approx S^{n-1}$
%   to $F=p^{-1}(b)$, giving us an element of $\pi_{n-1}(F,e)$. If $\tilde f$ and $\tilde
%   f'$ are different lifts, then we have the identity homotopy $p\circ \tilde f\simeq
%   p\circ \tilde f'$, which lifts to a homotopy $\tilde f\simeq \tilde f'$ which
%   restricts to a homotopy $\tilde f|_{\partial I^n}\simeq \tilde f'|_{\partial I^n}$, so
%   the map $\delta$ is well defined.
%
%   By construction, anything in the image of $\delta$ is homotopic to the constant map in
%   $E$ (via $\tilde f$), so $i_*\circ \delta$ is the constant map. Moreover, if $g:S^n\to
%   F$ is homotopic to the constant map in $E$ via $h:S^n\times I\to E$, then $p\circ
%   h$ is a map $S^{n+1}\to B$ whose image under $\delta$ is $g$. This proves exactness at
%   $\pi_n(F,e)$.
%
%   Finally, if $g:(I^n,\partial I^n)\to B$ such that $\delta(g)$ is homotopic to the
%   constant map via $h:\partial I^n \times I\to F$, then we can patch together $\tilde g$
%   and $h$ go get a map $(S^n,*)\to (E,e)$ which projects to $g$. This proves exactness
%   at $\pi_n(E,e)$.
% \end{proof}

 \section{Fiber bundles and Covering spaces}
 \begin{definition}
   A \emph{fiber bundle} is a surjective map $p:E\to B$ such that every $b\in B$ has an
   open neighborhood $U$ and a homeomorphism $f:U\times p^{-1}(b)\to p^{-1}(U)$ with
   $p\circ f$ equal to the projection onto the first coordinate. \xymatrix @ur{
    U\times p^{-1}(b)\ar[dr]_(.55)f^(.55)\approx \ar[d]_{p_1} \\ U & p^{-1}(U)\ar[l]^p
   }
 \end{definition}
 Note that $\xymatrix @-1pc @ur {E'\ar[d]_{p'} \ar[dr]^f \\ B & E\ar[l]^p}$ commutes if and
 only if $f\bigl((p')^{-1}(b)\bigr)\subseteq p^{-1}(b)$.
 \begin{lemma}
   If $B$ is connected, then all the fibers (inverse images of points) are homeomorphic.
 \end{lemma}
 \begin{proof}
   Let $b_0\in B$, then $\{b\in B|F_b\approx F_{b_0}\}$ and $\{b\in B|F_b\not\approx
   F_{b_0}\}$ are disjoint open sets covering $B$, and the first is non-empty, so it must
   be all of $B$.
 \end{proof}
 \begin{definition}
   A fiber bundle is called a \emph{covering} if the fibers are discrete.
 \end{definition}
 \begin{definition}
   A \emph{vector bundle} is a fiber bundle where each fiber comes with a vector space
   structure, and the local trivialization maps $f:U\times
   p^{-1}(b)\xrightarrow{\approx} p^{-1}(U)$ are required to be linear on fibers.
 \end{definition}
 For example, if $M$ is a smooth manifold, then the tangent bundle $TM$ is a vector
 bundle.
 \begin{definition}
   If $G$ is a topological group, then a \emph{principal $G$-bundle} is a fiber bundle
   where each fiber comes with a $G$-action, and we have local trivialization maps
   $f:U\times G\xrightarrow{\approx} p^{-1}(U)$ which are $G$-equivariant on fibers.
 \end{definition}
 \begin{example}
   $\ZZ\to \RR \to S^1$ is a principal $\ZZ$-bundle and a covering. $\ZZ/n\ZZ
   \hookrightarrow S^1 \to S^1$ is principal $\ZZ/n\ZZ$-bundle and a covering.
 \end{example}
 \section{Fiber bundles are Serre fibrations}
 \begin{lemma}\label{L:RelIsomorphismI^n}
   $\bigl(I^n\times I,I^n\times \{0\}\bigr) \approx \bigl( I^n\times I, I^n\times \{0\}
   \cup \partial I^n\times I\bigr)$. That is, there is a homeomorphism $I^n\times I\to
   I^n\times I$ sending $I^n\times \{0\}$ to $I^n\times \{0\} \cup \partial I^n\times I$.
   \qquad
   $\rule{.3pt}{12.2pt}\rule{12pt}{1pt}\llap{\rule[12pt]{12pt}{.3pt}}\rule{.3pt}{12.2pt}
   \approx
   \rule{1pt}{12.2pt}\rule{12pt}{1pt}\llap{\rule[12pt]{12pt}{.3pt}}\rule{1pt}{12.2pt}$
 \end{lemma}
 \begin{theorem}
   If $p:E\to B$ is a fiber bundle and $(X,A)$ is a relative CW-complex,\footnote{This
   means that $A$ is an arbitrary topological space, and $X$ is obtained from $A$ by
   attaching cells of increasing dimensions.} Then
   \[\xymatrix{
    X\times \{0\} \cup A \times I \ar@{}[d]|{\cap\rule{.3pt}{5pt}} \ar[r] & E\ar[d]^p\\
    X\times I\ar[r] \ar@{-->}[ur]_{\exists} & B
   }\]
    In particular, any fiber bundle is a Serre fibration. Moreover, if $p:E\to B$ is a
    covering, then the dashed arrow is unique.
 \end{theorem}
 \begin{proof}
   First we do the non-relative case (i.e.\ the case $A=\varnothing$).

   \underline{Step 1: Trivial bundles}: If $E=B\times F$, then use \quad
   $\xymatrix @C=4pc {
    X\times \{0\} \ar[r]^{(f,g)} \ar@{}[d]|{\cap\rule{.3pt}{5pt}} & B\times F\ar[d]^{p_1}\\
    X\times I \ar[r]^h \ar@{-->}[ur]|{(h,g\circ p_1)} & B }$ where $p_1$ always means
   ``project to the first coordinate''. Note that if $F$ is discrete, then $F^{X\times
   I}\approx (F^I)^X=F^X$, so $g\circ p_1$ is the unique map $X\times I\to F$ compatible
   with $g:X\times \{0\}\to F$.

   \marginpar{$\xymatrix{ \varnothing \ar[r] \ar[d] & E\ar[d]^p\\ \varnothing \ar[r]
   \ar@{-->}[ur] & B}$}
   \underline{Step 2: $X=I^n\approx D^n$}: We do this by induction
   on $n$. If $n=-1$, then $I^n=\varnothing$, so the statement is vacuous (see margin).
   Now assume we can lift $I^{n-1}\times I$, given the lift on $I^{n-1}\times \{0\}$ (and
   the lift is unique if $F$ is discrete). Let $\{U_i\}$ be a trivializing open cover of
   $B$, and let $\varepsilon$ be the Lebesgue number of the pull-back cover under the map
   $I^n\times I\to B$, so that the image of any little cube of side length $\varepsilon$
   lies entirely in one of the $U_i$. Now cut $I^n\times I$ into cubes of side length
   $\varepsilon$ and look at the following picture; the black and dark gray indicates
   part of the domain where we have lifted $I^n\times I\to B$.
   \[
   \raisebox{-.5cm}{
   \begin{pspicture}(-.2,-.5)(4.5,2)
     \psframe[linecolor=lightgray, fillstyle=crosshatch, hatchcolor=lightgray,
              hatchangle=0, hatchwidth=.2mm, hatchsep=3.8mm](0,0)(2,2)
     \psline[linewidth=.4mm](0,0)(2,0)
     \uput[180](0,1){$I$} \uput[-90](1,0){$I^n$}
     \psline{->}(2.5,1)(4,1) \uput[90](3.25,1){Induct} \uput[-90](3.25,1){on $n$}
   \end{pspicture}}
   \begin{pspicture}(0,0)(4.5,2)
     \psframe[linecolor=lightgray, fillstyle=crosshatch, hatchcolor=lightgray,
              hatchangle=0, hatchwidth=.2mm, hatchsep=3.8mm](0,0)(2,2)
     \psframe[linewidth=0pt, fillstyle=vlines,
              hatchangle=0, hatchwidth=.4mm, hatchsep=3.6mm](0,0)(2,2)
     \psline[linewidth=.4mm](0,2)(0,0)(2,0)(2,2)
     \psline{->}(2.5,1)(4,1) \uput[90](3.25,1){Lemma} \uput[-90](3.25,1){+Step 1}
   \end{pspicture}
   \begin{pspicture}(0,0)(4.5,2)
     \psframe[linecolor=lightgray, fillstyle=crosshatch, hatchcolor=lightgray,
              hatchangle=0, hatchwidth=.2mm, hatchsep=3.8mm](0,0)(2,2)
     \psframe[linewidth=0pt, fillstyle=vlines,
              hatchangle=0, hatchwidth=.4mm, hatchsep=3.6mm](0,0)(2,2)
     \psline[linewidth=.4mm](0,2)(0,0)(2,0)(2,2)
     \psframe[fillstyle=solid, fillcolor=gray](0,0)(.4,.4)
     \psline{->}(2.5,1)(4,1) \uput[90](3.25,1){Induct}
   \end{pspicture}
   \begin{pspicture}(0,0)(2,2)
     \psframe[fillstyle=solid, fillcolor=gray](0,0)(2,2)
     \psframe[fillstyle=crosshatch, hatchangle=0, hatchwidth=.2mm, hatchsep=3.8mm](0,0)(2,2)
   \end{pspicture}
   \]
   By induction on $n$, we can lift on a grid of $I^{n-1}\times I$, cutting $I^n\times I$
   into narrow columns of width $\varepsilon$. By Lemma \ref{L:RelIsomorphismI^n},
   lifting the $\varepsilon$-size $I^n\times I$ given the lift on $I^n\times \{0\}\cup
   \partial I^n \times I$ is the same as lifting $I^n\times I$ given the lift on
   $I^n\times \{0\}$, which we know how to do by Step 1 because the bundle is trivial
   over each $U_i$. If $F$ is discrete, then all the intermediate lifts are unique, so
   the lift is unique.

   \underline{Step 3: $X$ a CW complex}: If $X$ is an arbitrary CW complex, assume by
   induction that we've lifted $h:X\times I\to B$ to $\tilde h:X^{(n-1)}\times I\to E$.
   \[\xymatrix{
   \coprod S^n\times I \cup D^n\times \{0\} \ar[r] \ar[d] & E\ar[d]\\
   \coprod D^n\times I \ar@{-->}[ur] \ar[r] & B
   }\qquad \xymatrix{
    \coprod S^n\times I \ar[r] \ar[d] \ar@{}[dr]|(.75)\po &
    X^{(n-1)}\times I \ar[d] \ar@/^/[dr]^(.6){\txt{already\\ have}}\\
    \coprod D^n\times I \ar[r] \ar@/_1pc/[rr]_{\txt{given by Step 2}}
    & X^{(n)}\times I \ar@{-->}[r]^(.6){\exists !} & E
   }\]
   We use Step 2 and Lemma
   \ref{L:RelIsomorphismI^n} (keeping in mind that $D^n\approx I^n$) to get the dashed
   arrow in the left diagram, which is the bottom curved arrow in the right diagram. By
   the universal property of push-out, we get a lift $X^{(n)}\times I \to E$. Now we use
   the usual limiting argument for CW complexes to get a lift $X\times I \to E$. As
   usual, it is clear that uniqueness holds if $F$ is discrete.

   \medskip
   Note that in Step 3, we didn't use anything about $X^{(n-1)}$; it could have been
   completely general, so the proof applies to the relative case.
 \end{proof}

 \begin{corollary}
   Fiber bundles have the long exact sequence of homotopy groups described in Theorem
   \ref{C:fibrationLES}.
 \end{corollary}
 \begin{corollary}[to the Corollary]
   If $F$ is discrete, then $p_*:\pi_n(E,e)\xrightarrow{\sim} \pi_n(B,b)$ for $n>1$.
   Moreover, if $E$ is connected, then $\pi_1(E)\hookrightarrow \pi_1(B)$ has index
   $|F|$.
 \end{corollary}
 Note that if $F$ is discrete and $E$ is connected, then the connecting map $\delta:
 \pi_1(B,b)\to \pi_0(F,e)=F$ is surjective, yielding a transitive action of $\pi_1(B,b)$
 on $F$.

 \section{Long cofibration and fibration sequences}
 We denote the suspension of $X$ by $\Sigma X=X\times I/\sim$, with $(x,1)\sim (x',1)$
 and $(x,0)\sim (x',0)$. We denote the reduced suspension by $SX=S^1\wedge X$; this is
 the suspension, with the additional identification $(x_0,s)\sim (x_0,t)$ for some base
 point $x_0\in X$. Note that $\Sigma X\simeq SX$.
 \begin{lemma}
   If $X$ and $Y$ are compact, then $X\wedge Y$ is the one point compactification of
   $(X\smallsetminus \{x\})\times (Y\smallsetminus \{y\})$.
 \end{lemma}
 In particular $S^n\wedge S^m\approx S^{n+m}$ because $\RR^n\times \RR^m\approx
 \RR^{n+m}$.
 \begin{lemma}
   If $i:A\to X$ is a cofibration, then there is a long cofibration sequence (with maps
   defined up to homotopy)
   \[
    A\xrightarrow{i} X\to X/A \to \Sigma A \xrightarrow{\Sigma i} \Sigma X\to \Sigma
    (X/A) \to \Sigma^2 A\xrightarrow{\Sigma^2 i}\cdots.
   \]
   That is, any three consecutive terms are \emph{homotopic} to a cofibration triple.
 \end{lemma}
 \begin{proof}
   We know that $A\to X\to X/A$ is a cofibration. We have $X/A\simeq X\cup_{A} C(A)$,
   where $C(A)$ is the cone on $A$. Then it is clear that $X\to X\cup_A C(A)\to \Sigma A$
   is a cofibration. Similarly, $\Sigma A \simeq \bigl(X\cup_A C(A)\bigr)\cup_X C(X)$,
   and $\bigl(X\cup_A C(A)\bigr)\to \bigl(X\cup_A C(A)\bigr)\cup_X C(X) \to \Sigma X$ is
   a cofibration. Finally, $\Sigma$ is a functor which preserves cofibrations, so we're
   done.
   \[\raisebox{5pc}{\xymatrix @R=.3pc{
   A \ar@{}[d]|\parallel & X \ar@{}[d]|\parallel & X/A \ar@{}[d]|(.4){|\wr} & \Sigma A
   \ar@{}[d]|(.4){|\wr} & \Sigma
   X \ar@{}[d]|\parallel\\
   \rule[1pt]{.5cm}{1pt} \ar[r] & \rule[1pt]{1cm}{.4pt} \llap{\rule[.7pt]{.5cm}{1pt}}
   \ar[r] & \raisebox{8mm}{\begin{pspicture}(0,0)(1,.75)
     \psline[linewidth=.4pt](0,0)(1,0) \psline[linewidth=1pt](.5,0)(1,0)(.75,.5)(.5,0)
   \end{pspicture}} \ar[r] &
   \begin{pspicture}(0,-1)(1,.75)
     \psline[linewidth=.4pt](0,0)(1,0)(.5,-1)(0,0) \psline[linewidth=1pt](.5,0)(1,0)(.75,.5)(.5,0)
     \psline[linewidth=.4pt](.1,-.2)(.9,-.2)
   \end{pspicture} \ar[r] &
   \raisebox{-17.5mm}{\begin{pspicture}(0,-.8)(.8,.3)
     \psline[linewidth=.4pt](0,0)(.8,0)(.4,.3)(0,0)(.4,-.8)(.8,0) \psdots(.4,.3)
   \end{pspicture}}
   }}\qedhere\]
 \end{proof}
 \begin{corollary}
   For any pointed space $Y$, by Lemma \ref{Lcofibrationexact}, we have a long exact
   sequence
   \[
    \cdots \to[S(X/A),Y]_{pt} \to[SX,Y]_{pt} \to[S A,Y]_{pt} \to[X/A,Y]_{pt} \to[X,Y]_{pt} \to[A,Y]_{pt}
   \]
 \end{corollary}
 \begin{remark}
   This sequence is called a \emph{Puppe sequence}. If we take $Y=K(A,n)$, then we get
   the long exact sequence in cohomology for the pair $(X,A)$. To see this, note that
   $\pi_{i+1}(X)=[S^{i+1},X]=[S^i,\W X]=\pi_i(\W X)$. In particular, $\W
   K(A,n)=K(A,n-1)$, so $[S^k X,K(A,n)]=[X,\W^k
   K(A,n)]=[X,K(A,n-k)]=H^{n-k}(X,A)$.\anton{clean}
 \end{remark}

 ``Dually'', we get
 \begin{lemma}
   If $F\to E\to B$ is a fibration, then there is a long fibration sequence
   \[
    \cdots\to \W^3 B \to \W^2 F\to \W^2 E\to \W^2 B\to \W F\to \W E\to \W B\to F\to E\to B
   \]
   That is, any three consecutive terms are \emph{homotopic} to a fibration triple.
 \end{lemma}
 \begin{proof}
   \anton{do}
 \end{proof}
 \begin{corollary} \label{C:fibrationLES}
   For any pointed space $Y$, by Lemma \ref{L:fibrationexact}, we get the long exact
   sequence
   \[
    \cdots\to [Y,\W^2 E]\to [Y,\W^2 B]\to [Y,\W F]\to [Y,\W E]\to [Y,\W B]\to [Y,F\to E]\to
    [Y,B]
   \]
 \end{corollary}
 In particular, taking $Y=S^0$, and noting that $[S^0,\W^n X]=[S^n,X]$, we get the long
 exact sequence in homotopy groups.\anton{clean?}

 \section{Some equivalences of categories}
 We'll prove the following theorem later. $\textbf{hCW}(n)$ is the category of homotopy
 classes of CW complexes $X$ with $\pi_k(X)=0$ for $k\neq n$.
 \begin{theorem}
   $\pi_n$ is an equivalence of categories $\textrm{\bf hCW}(n)\to\textrm{\bf Ab}$, with
   inverse $K(-,n)$.
 \end{theorem}
 In particular, $K(A,n)$ is well-defined up to homotopy is always an H-group.
 \begin{definition}
   A functor $F:\C\to \D$ is an \emph{equivalence of categories} if there is a functor
   $G:\D\to \C$ so that $F\circ G$ (resp.~$G\circ F$) is naturally isomorphic to $\id_\D$
   (resp.~$\id_\C$).
 \end{definition}
 \begin{lemma}
   A functor $F:\C\to \D$ is an equivalence of categories if and only if it is
   essentially surjective (the image contains an element from each isomorphism class in
   $\D$) and for every pair of objects $c_0$ and $c_1$ of $\C$, $F$ induces a bijection
   $\hom_\C(c_0,c_1)\xrightarrow{\sim} \hom_\D(Fc_0,Fc_1)$.
 \end{lemma}
 \begin{proof}
   \anton{}
 \end{proof}
 Note that this lemma constructs the inverse functor.

 \begin{lemma}
   If $F\to E\to B$ is a covering, then $\pi_1(B)$ acts on $F$.
 \end{lemma}
 \begin{proof}
   \anton{follows from the LES}
 \end{proof}
 \begin{lemma}
   If $B$ is path connected, locally path connected, and semi-locally simply
   connected,\footnote{For every point $x\in B$ and every neighborhood $V$ of $x$, there
   is a neighborhood $U\subseteq V$ of $x$ so that any loop in $U$ is homotopic to a
   constant loop \emph{in $B$} (the homotopy may go outside of $U$). A connected CW
   complex will have all these connectivity conditions.} then $B$ has a universal
   covering space.
 \end{lemma}
 \begin{proof}[Not a Proof]
   Define $\tilde E$ to be the set of paths in $B$ starting at $b$ modulo homotopy rel
   endpoints. Then we have a map $\tilde E\to B$, sending a path to its other endpoint.

   Even if $B$ satisfies the conditions of the Lemma, this is not always a covering.
   Sometimes you have to change the topology on $\tilde E$ to make it the universal
   cover. If $B$ is metrizable, then $\tilde E$ does the trick as is.
 \end{proof}
 Let $\textbf{Cov}(B)$ be the category of covering spaces of $B$, and let
 $G$-\textbf{Set} be the category of sets with a $G$ action (and morphisms are
 equivariant set maps).
 \begin{theorem}
   If $B$ is path connected, locally path connected, and semi-locally simply connected,
   then there is an equivalence of categories $\textrm{\bf Cov}(B)\to
   \pi_1(B)\textrm{-\bf Set}$, given by $E\mapsto F$.
 \end{theorem}
 \begin{proof}[``Proof'']
   The inverse functor is given by $F\mapsto \tilde E\times_{\pi_1(B)} F$.
 \end{proof}
 \begin{remark}
   Any bundle $E\to B$ with fiber $F$ is isomorphic to $P\times_G F$ for some group
   $G\subseteq \text{Homeo}(F)$ and some principal $G$ bundle $P\to B$. Thus, if you
   understand principal bundles, you understand all bundles.
 \end{remark}
 \begin{corollary}
   If $B$ is as above, then $E$ is connected if and only if the $\pi_1(B)$ action on $F$
   is transitive.
 \end{corollary}
 \begin{proof}
   The connected coverings of $B$ are exactly the ones which cannot be written as
   coproducts (disjoint unions) of others. The $\pi_1(B)$-sets that cannot be written as
   coproducts (disjoint unions) of others are exactly those for which the action is
   transitive.
 \end{proof}
 Note that if $E$ is connected, then the action of $\pi_1(B)$ on $F$ is transitive, so
 $F\cong \pi_1(B)/U$, where $U$ is the subgroup of $\pi_1(B)$ that stabilizes $e_0\in F$.


 $\rho:G\to GL(V)$ a representation, and $P\to B$ a principal $G$-bundle, then $P\times_G
 V\to B$ is a vector bundle! \anton{}

 \begin{corollary}
   $E\to B$ has a section if and only if $F$ has a $\pi_1(B)$-fixed point. In fact, there
   is a bijection between sections and fixed points.
 \end{corollary}
 \begin{proof}
   A section is a bundle map from the trivial bundle $B\xrightarrow{\id} B$ to $E\to B$.
   Under the equivalence of categories, this is a $\pi_1(B)$-equivariant map from the one
   point $\pi_1(B)$-set to $F$, i.e.~a fixed point.
 \end{proof}
 \begin{corollary}
   If $\pi_1(B)$ is trivial, then all covers of $B$ are trivial.
 \end{corollary}
 \begin{corollary}
   Isomorphism classes of connected covers of $B$ are in bijection with subgroups of
   $\pi_1(B)$ up to conjugation.
 \end{corollary}
 \begin{proof}
   The first is $\pi_0\bigl(\textbf{Cov}_\text{conn}(B)\bigr)$, and the second is
   $\pi_0\bigl( \pi_1(B)\text{-}\textbf{Set}_\text{trans}\bigr)$. These sets are in
   bijection by the equivalence of categories.
 \end{proof}
 \begin{corollary}
   Isomorphism classes of connected covers of $B$ (with base point) are in bijection with
   subgroups of $\pi_1(B)$.
 \end{corollary}
 The subgroup corresponding to $E\to B$ is the stabilizer of the base point. This is
 sometimes called ``Galois theory for coverings''.

 \section{Finally Homology}
 We'll define simplicial homology on $\Delta$-sets, then singular homology, which will
 work for all topological spaces, but will be monstrous. Then we'll define cellular
 homology, which is the awesomest homology theory for calculating (but only for CW
 complexes). We won't cover \v Cech cohomology or its generalization, sheaf cohomology.

 A homology theory is usually defined by using a topological space to produce a
\emph{chain complex}, a sequence of abelian groups
 \[
    \cdots \xrightarrow{d_{n+2}} C_{n+1}\xrightarrow{d_{n+1}} C_n\xrightarrow{d_n} C_{n-1} \xrightarrow{d_{n-1}}\cdots
 \]
 such that $d_n\circ d_{n+1}=0$ for all $n$. We call the $d_n$ \emph{boundary operators}.
 We define the $n$-th \emph{homology} of the chain complex to be $\ker d_{n}/\im
 d_{n+1}$, and call this the $n$-th homology of the topological space we started with.

 \subsection*{Relation to homotopy}
 We seek to define functors $H_n:\textbf{hTop}\to \textbf{Ab}$ which satisfy some nice
 properties (the Eilenberg-Steenrod axioms). From these properties, one can prove that
 $H_n(S^n)\cong \ZZ$. This allows us to define the \emph{Hurewicz map} $\phi:\pi_n(X)\to
 H_n(X)$. Given some $f:S^n\to X$, we get an induced map $f_*:H_n(S^n)\cong \ZZ\to
 H_n(X)$. We define $\phi([f])$ to be $f_*(1)\in H_n(X)$.
 \begin{theorem}[Hurewicz]
   $H_0(X)$ is the free abelian group on $\pi_0(X)$. If $X$ is 0-connected, $H_1(X)$ is
   the abelianization of $\pi_1(X)$. If $X$ is $(n-1)$-connected with $n\ge 2$, then the
   Hurewicz map $\pi_n(X)\to H_n(X)$ is an isomorphism.
 \end{theorem}
 In fact, there is a (stronger) relative version.
 \begin{theorem}[Hurewicz]
   If $f:X\to Y$ induces isomorphisms on $\pi_i$ for $i\le n$, then it induces
   isomorphisms on $H_i$ for $i\le n$. If $X$ and $Y$ are 1-connected, then the converse
   is true.
 \end{theorem}

 \section{Simplicial Homology}

% Poincar\'e defined homology about 100 years ago (before the definition of a topological
% space!) using Riemann surfaces and polyhedra. Roughly, given a polyhedron, we'd like to
% build a chain complex out of it with

 A \emph{simplicial complex} is a set obtained by gluing together simplices along their
 faces. An $n$-dimensional simplex has $n+1$ faces which are subsimplices. We can encode
 the gluing information in a \emph{$\Delta$-set}.
 \begin{definition}
   A \emph{$\Delta$-set} is a sequence of sets $S_n$, with maps $d^n_i:S_n\to S_{n-1}$
   for $0\le i\le n$ that satisfy the relation $d_j\circ d_i = d_{i-1}\circ d_j$ for
   $j<i$.
 \end{definition}
 From a $\Delta$-set, we can produce a \emph{geometric realization} $|S_\bullet|$ as in
 the homework.

 Some facts:
 \begin{itemize}
   \item \underline{Whitehead's Theorem}: If $X$ and $Y$ are CW complexes and $f:X\to Y$
   is a \emph{weak equivalence} ($f$ induces isomorphisms $\pi_n(X)\to \pi_n(Y)$ for all
   $n$), then $f$ is a homotopy equivalence.
   \item Any CW complex is homotopy equivalent to some $|S_\bullet|$.
   \item Given \emph{any} topological space $X$, there is a canonical CW complex $X'$ and
   a weak homotopy equivalence $X'\to X$.
 \end{itemize}

 Given a $\Delta$-set $S_\bullet$, we define a chain complex by setting $C_n=\ZZ\cdot
 S_n$, the free abelian group on the ``$n$-simplices'', with $d_n = \sum_{i=0}^n (-1)^i
 d^n_i$, where the maps $d^n_i$ have been extended linearly. It is a standard exercise to
 check that $d_{n}\circ d_{n-1}=0$. We define the \emph{simplicial homology} of
 $S_\bullet$ (or $|S_\bullet|$) to be the homology of this chain complex.

 It is true but not at all obvious that if $|S_\bullet|\approx |T_\bullet|$, then
 $S_\bullet$ and $T_\bullet$ produce the same homology groups. One can show that if
 $T_\bullet$ is a refinement of $S_\bullet$, then this holds. For a long time, people
 conjectured that any two simplicial decompositions of a space, there exists a common
 refinement. This conjecture, called the Hauptvermutung, would prove the result. However,
 the Hauptvermutung is false! This was demonstrated by Milnor around 1960.

 \section{Singular Homology}
 The standard $n$-simplex is $\Delta^n = \{\mathbf{x}\in \RR^{n+1}|x_i\ge 0\text{ for all
 $i$ and }\sum x_i=1\}$.
 \begin{definition}
   The singular $\Delta$-set $\Delta_\bullet(X)$ of a topological space $X$ has
   $\Delta_n(X)=\{\text{continuous maps }\sigma:\Delta^n\to X\}$ with the obvious
   boundary maps.
 \end{definition}
 \begin{definition}
   The \emph{singular homology} of a topological space $X$ is the simplicial homology of
   $\Delta_\bullet(X)$.
 \end{definition}
 Thus, $H_n$ is the composition of functors
 \[
    H_n: \textbf{Top}\xrightarrow{\Delta_\bullet} \Delta\text{-}\textbf{Set}
    \xrightarrow{\text{``free''}} \textbf{Chain} \xrightarrow{\text{$n$-th homology}}
    \textbf{Ab}.
 \]
 We will denote by $S_*(X)$ the singular chain complex $C_*(\Delta_\bullet(X))$ of $X$.
 \begin{lemma}
   $H_0(X)=\ZZ\cdot \pi_0(X)$.
 \end{lemma}
 \begin{proof}
   $H_0(X) = \coker\bigl(\ZZ[\sigma:I\to X]\xrightarrow{d} \ZZ[pt\to X]\bigr)$, where
   $d(\sigma)=\sigma(1)-\sigma(0)$. Define a map $H_0(X) = S_0(X)/\im d \to \ZZ\cdot
   \pi_0(X)$ by sending $x\in X$ (viewed as a map $pt\to X$) to the connected component
   of $x$. The map is clearly surjective. If there is some linear combination in the
   kernel, then each path component has as many pluses as minuses, so you can match them
   up with paths, so that combination is in the image of $d$. Thus, the map is an
   isomorphism.
 \end{proof}
 \begin{example}\label{dimensionAxiom}
   If $X$ is a point, then $S_i(X)=\ZZ$ for each $i\ge 0$ because there is only one map
   from $\Delta^i$ to a point. $d^n$ is an alternating sum of $n+1$ identity maps, so the
   singular chain complex is
   \[\cdots
    \xrightarrow{0} \overset{S_4}{\ZZ}
    \xrightarrow{1} \overset{S_3}{\ZZ}
    \xrightarrow{0} \overset{S_2}{\ZZ}
    \xrightarrow{1} \overset{S_1}{\ZZ}
    \xrightarrow{0} \overset{S_0}{\ZZ}
   \]
   so $H_n(X) = \begin{cases}
     \ZZ & n=0\\
     0 & n\neq 0
   \end{cases}$
 \end{example}
 \begin{definition}
   An $n$-dimensional manifold $M$ is \emph{orientable}\footnote{This is really the
   definition of $\ZZ$-orientable. See Definition \ref{Def:orient} for a more general
   definition.} if $H_n(M)\cong \ZZ$. In this case, an \emph{orientation} of $M$ is a
   choice of generator for $H_n(M)$.
 \end{definition}
 \begin{theorem}\label{T:DeltaGeomAdjunction}
   There is a natural adjunction
   \[
    \hom_\textrm{\bf hTop}(|S_\bullet|,X) \cong \hom_{\Delta\textrm{-\bf Set}}\bigl(S_\bullet,
    \Delta_\bullet (X)\bigr).
   \]
   The natural map $S_\bullet\to \Delta_\bullet( |S_\bullet|)$ induces isomorphisms on
   homology, and the natural map $|\Delta_\bullet(X)|\to X$ is a weak equivalence.
 \end{theorem}
 It follows that the simplicial homology of $S_\bullet$ is equal to the singular homology
 of $|S_\bullet|$, a handy fact for calculation. Also, for any space $X$, we get a CW
 complex $X':=|\Delta_\bullet(X)|$ and a weak equivalence $X'\to X$.

 \section{Eilenberg-Steenrod Axioms}
 It turns out that any sequence of functors satisfying the following axioms are naturally
 isomorphic to singular homology. In practice, one can use these axioms to compute
 homology. We will have to prove that singular homology satisfies these axioms to verify
 that such functors exist.
 \begin{enumerate}
   \item (Homotopy axiom) The functors $H_n:\textbf{Top}\to \textbf{Ab}$ factor through
   \textbf{hTop}.
   \item \marginpar{\hspace*{-1em}
     $\xymatrix @-.5pc @dr{U\cap V \ar[d]^{j_U} \ar[r]_{j_V} & V
     \ar[d]_{i_V}\\ U \ar[r]^{i_U} & U\cup V}$}
   (Mayer-Vietoris axiom) If $U, V\subseteq X$ are open sets, then there is a
   natural long exact sequence
   \[
    \cdots \to H_n(U\cap V)\xrightarrow{(j_U,j_V)} H_n(U)\oplus
    H_n(V)\xrightarrow{i_U-i_V} H_n(U\cup V)\xrightarrow{\ \delta\ }H_{n-1}(U\cap
    V) \to \cdots
   \]
   \item (Dimension axiom) $H_0(pt)=\ZZ$ and $H_n(pt)=0$ for $n>0$.
   \item (Additivity axiom) The inclusions $X_\alpha\hookrightarrow \coprod X_\alpha$
       induce an isomorphism $\bigoplus H_n(X_\alpha) \xrightarrow{\sim}
       H_n\bigl(\coprod X_\alpha\bigr)$.
   \item (Weak homotopy axiom) A weak equivalence induces an isomorphism on homology.
 \end{enumerate}

 For any homology $H_n$, we can define \emph{reduced homology} $\H_n(X) := \ker \bigl(
 H_n(X)\to H_n(pt)\bigr)$. It will satisfy the following axioms:
 {\footnotesize
 \begin{enumerate}
   \item (Homotopy axiom) $\H_n:\textbf{Top}\to \textbf{Ab}$ factor through
   \textbf{hTop}.
   \item (Mayer-Vietoris axiom) If $U, V\subseteq X$ are open sets, then there is a
   natural long exact sequence
   \[
    \cdots \to \H_n(U\cap V)\xrightarrow{(j_U,j_V)} \H_n(U)\oplus
    \H_n(V)\xrightarrow{i_U-i_V} \H_n(U\cup V)\xrightarrow{\ \delta\ }\H_{n-1}(U\cap
    V) \to \cdots
   \]
   \item (Dimension axiom) $\H_n(pt)=0$ for all $n$.
   \item (Additivity axiom) The inclusions $X_\alpha\hookrightarrow \bigvee X_\alpha$
       induce an isomorphism $\bigoplus \H_n(X_\alpha)\xrightarrow{\sim}\H_n\bigl(\bigvee
       X_\alpha\bigr)$.
   \item (Weak homotopy axiom) A weak equivalence induces an isomorphism on reduced homology.
 \end{enumerate}}

 \begin{lemma}
   $\H_n(S^n)=\ZZ$ and $\H_k(S^n)=0$ for $k\neq n$.
 \end{lemma}
 \begin{proof}
   Induct on $n$. It is true for $n=0$ by the dimension and additivity axioms (for
   $H_n$). For $n\ge 1$, write $S^n=U\cap V$, with $U\simeq V\simeq \ast$ and $U\cap
   V\simeq S^{n-1}$. By the Mayer-Vietoris property, we get
   \[
    \cdots \to \overbrace{\H_k(U)\oplus \H_k(V)}^0 \to \H_k(S^n)\xrightarrow{\sim}
    \H_{k-1}(\overbrace{U\cap V}^{\simeq S^{n-1}}) \to \overbrace{\H_{k-1}(U)\oplus \H_{k-1}(V)}^0 \to \cdots
   \]
   which proves the result.
 \end{proof}
 \begin{proposition}
   Singular homology satisfies the weak homotopy axiom.
 \end{proposition}
 \begin{proof}
   If $X\to Y$ is a weak equivalence, it induces isomorphisms on all homotopy groups, so
   by the relative Hurewicz theorem, it induces isomorphisms on all homology groups.
 \end{proof}
 \begin{proposition}
   Singular homology satisfies the additivity axiom.
 \end{proposition}
 \begin{proof}
   This follows from the fact that the functors $\Delta_\bullet$, $C_*$, and $n$-the
   homology of a chain all preserve coproducts.
   \def\copd{{\textstyle\coprod}}
   \begin{gather*}
     \Delta_n(\copd X_\alpha) = \{\sigma:\Delta^n\to \copd
     X_\alpha\}=\copd\{\sigma:\Delta^n\to X_\alpha\} = \copd \Delta_n(X_\alpha)\\
     C_n(\copd S_{\bullet,\alpha}) = \ZZ\cdot \copd S_{n,\alpha} = \bigoplus \ZZ\cdot
     S_{n,\alpha} = \bigoplus C_n(S_{\bullet,\alpha})\\
     H_n\bigl( \bigoplus C_{*,\alpha}\bigr)  = \bigoplus H_n(C_{*,\alpha})
   \end{gather*}
 \end{proof}
 We've already verified the dimension axiom in Example \ref{dimensionAxiom}.

 \begin{lemma} \label{L:SEStoLES}
   Given a short exact sequence of chain complexes $0\to A_*\xrightarrow{f}
   B_*\xrightarrow{g} C_*\to 0$, there is a long exact sequence of homology groups
   \[
   \cdots \to H_n(A_*)\xrightarrow{f_*} H_n(B_*)\xrightarrow{g_*} H_n(C_*)\xrightarrow{\delta}
   H_{n-1}(A_*)\to \cdots
   \]
 \end{lemma}
 \begin{proof}
   A diagram chase.
 \end{proof}
 \begin{proposition}[Mayer-Vietoris for $\Delta$-sets]
   If $S_\bullet$ is a $\Delta$-set with sub-$\Delta$-sets $U_\bullet$ and $V_\bullet$,
   then
   \[
    0\to C_*(U_\bullet\cap V_\bullet)\xrightarrow{(j_U,j_V)} C_*(U_\bullet)\oplus
    C_*(V_\bullet) \xrightarrow{i_U-i_V} C_*(U_\bullet\cup V_\bullet)\to 0
   \]
   is a short exact sequence. In particular, since homology preserves direct sums, you
   get a long exact sequence in homology by the lemma.
 \end{proposition}
 \begin{proposition}
   Singular homology satisfies the Mayer-Vietoris axiom.
 \end{proposition}
 \begin{proof}
   It is clear that we get the exact sequence
   \[
    0\to S_*(U\cap V)\to S_*(U)\oplus S_*(V)\to S_*(U\cup V)
   \]
   The result follows from the following claim.
   \begin{claim}
     The inclusion $S_*(U)+ S_*(V)\hookrightarrow S_*(U\cup V)$ is a homotopy equivalence.
   \end{claim}
    To see the claim, consider the chain map $b:S_*(U\cup V)\to S_*(U\cup V)$ given by
    barycentric subdivision. We get that $b\simeq \id$ \anton{is there an easy way to see
    it?}, and for any simplex $\sigma$, $b^N(\sigma)\in S_*(U)+ S_*(V)$ for large enough
    $N$. \anton{this somehow works out}
 \end{proof}

 The hardest axiom to verify is the homotopy axiom. We will do it after we develop
 cellular homology.

 In the process, we will show that the Eilenberg-Steenrod axioms
 completely determine homology. We will define cellular homology in terms of a homology
 homology and then show that the cellular homology is isomorphic to the original
 homology. Then we will show that cellular homology is independent of the homology theory
 you started with.

 \section{Cellular Homology and Uniqueness of Homology}
 \begin{definition}
   If $A\subseteq X$, the \emph{relative singular homology} $H_n(X,A)$ is the $n$-th
   homology of the chain complex $S_*(X)/S_*(A)$.
 \end{definition}
 \begin{lemma}
   If $A\subseteq X$ is a cofibration, then $H_n(X,A)\cong \H_n(X/A)$.
 \end{lemma}
 \begin{proof}
   Note that $S_*(X)/S_*(A) \cong \widetilde S_*(X)/\widetilde S_*(A)$. By Lemma
   \ref{L:SEStoLES}, we get the long exact sequence shown as the top row in the diagram.
   \[\xymatrix{
     \H_n(A)\ar[r] \ar[d]^\id & \H_n(X)\ar[r]\ar[d]^\id & H_n(X,A) \ar[r]\ar[d]^\phi & \H_{n-1}(A) \ar[d]^\id\ar[r]& \H_{n-1}(X)\ar[d]^\id \\
     \H_n(A)\ar[r] & \H_n(X)\ar[r] & \H_n(X/A) \ar[r] & \H_{n-1}(A) \ar[r]& \H_{n-1}(X)
   }\]
   To get the bottom exact sequence, observe that $X/A \simeq X\cup_A CA$, so we may
   choose open sets $U,V\subseteq X\cup_A CA$ so that $U\simeq X$ and $V\simeq \ast$.
   Then the bottom row is just the Mayer-Vietoris sequence in reduced homology.

   The map $\phi$ is induced by the obvious map $\widetilde S_*(X)/\widetilde S_*(A) \to
   \widetilde S_*(X/A)$ and makes the above diagram commute. By the 5-lemma, we get the
   desired result.
 \end{proof}
 \begin{definition}
   For a CW complex $X$, we define the group of cellular $n$-chains to be
   $C_n(X)=H_n(X^{(n)},X^{(n-1)})$. The boundary map is defined as the composition
   $d^{CW}:H_n(X^{(n)},X^{(n-1)})\to H_{n-1}(X^{(n-1)})\to \H_n(X^{(n-1)},X^{(n-2)})$. The
   \emph{cellular homology} $H_n^{CW}(X)$ is the $n$-th homology of this chain complex.
 \end{definition}
 Note that the lemma shows that $C_n(X) = H_n(X^{(n)}/X^{(n-1)}) = \H_n(\bigvee_{I_n}
 S^n) \cong \ZZ \cdot I_n$, where $I_n$ indexes the $n$-cells of $X$.

 To prove that cellular homology agrees with singular homology, we'll need a few lemmas.

 \begin{lemma}\label{L:lemm1}
   $H_k(X^{(n)})=0$ for $k>n$.
 \end{lemma}
 \begin{proof}
   Induct on $n$. For $n=0$, it is true by additivity because $X^{(0)}$ is a collection
   of points. In general, we get an exact sequence
   \[
    \underbrace{H_k(X^{(n-1)})}_{0\text{ by induction}} \to H_k(X^{(n)}) \to
    \underbrace{H_k(X^{(n)},X^{(n-1)})}_{\cong \H_k(X^{(n)}/X^{(n-1)}) = \H_n(\bigvee
    S^n)=0 \hspace*{-5em}}
   \]
   so the middle term is zero.
 \end{proof}
 \begin{lemma}\label{L:lemm2}
   The inclusion $i:X^{(k)}\to X$ induces isomorphisms $H_n(X^{(k)})\to H_n(X)$ for $n<
   k$.
 \end{lemma}
 \begin{proof}
   We have the sequence
   \[
    \underbrace{H_{n+1}(X^{(n+\ell+1)},X^{(n+\ell)})}_0 \to H_n(X^{(n+\ell)}) \xrightarrow{i}
    H_n(X^{(n+\ell+1)}) \to \underbrace{H_n(X^{(n+\ell+1)},X^{(n+\ell)})}_0 .
   \]
   It follows that for $\ell>0$, inclusion induces isomorphisms $H_n(X^{(n+\ell)})\cong
   H_n(X^{(n+\ell+1)})$.

   To show that the inclusion $X^{(k)}\to X$ induces isomorphisms, we can take two
   approaches:
   \begin{enumerate}
     \item[(Bad)] We know that singular homology is represented by maps
     $\sigma:\Delta^n\to X$. Since $\Delta^n$ is compact, the image of $\sigma$ lies in a
     finite skeleton. This shows that $H_n(X^{(k)})\to H_n(X)$ is surjective (given the
     isomorphisms already constructed). Similarly, if some $\sigma$ is a boundary of
     $\tilde\sigma$ in $X$, then the image of $\tilde\sigma$ is in some finite skeleton,
     showing that the map is injective. This approach is bad because it explicitly uses
     the definition of singular homology.

     \item[(Good)] From the Eilenberg-Steenrod axioms, one can prove that homology
     respects filtered colimits. This is the good version because it proves the lemma for
     arbitrary homology theories.\qedhere
   \end{enumerate}
 \end{proof}
 \begin{theorem}
   Cellular homology is naturally isomorphic to singular homology.
 \end{theorem}
 \begin{proof} In the following diagram, all the horizontal and vertical sequences are
 exact. By the way, this makes it clear that $(d^{CW})^2=0$.
   \[\small \xymatrix @C-1pc{
    & &
   **[l] 0\overset{\ref{L:lemm1}}= H_n(X^{(n-1)}) \ar[d] & \fbox{$H_n(X)$}\ar@{}[d]|{\parallel}^{\wr\: \ref{L:lemm2}} & \H_n(\bigvee S^{n+1})=0 \ar@{}[d]|{\parallel}^{\wr} \\
   H_{n+1}(X^{(n+1)}) \ar[r] & H_{n+1}(X^{(n+1)},X^{(n)}) \ar[rd]_<>(.5){d^{CW}} \ar[r]^<>(.5)\delta &
   H_n(X^{(n)}) \ar[d]^{j_n} \ar[r] & H_n(X^{(n+1)}) \ar[r] & H_n(X^{(n+1)},X^{(n)})\\
    & C_{n+1}(X) \ar@{}[u]|{\parallel} & H_n(X^{(n)},X^{(n-1)}) \ar[d]_\delta \ar@{}[r]|<>(.5){=} \ar[rd]^{d^{CW}} & C_n(X) \\
    & 0\overset{\ref{L:lemm1}}{=} H_{n-1}(X^{(n-2)})\ar[r] & H_{n-1}(X^{(n-1)})
   \ar[r]_<>(.5){j_{n-1}} & \rlap{$H_{n-1}(X^{(n-1)},X^{(n-2)})$}\qquad
   }\]
   Now we compute the $H_n(X)$ which is boxed.
   \begin{align*}
     H_n(X) &\cong H_n(X^{(n)})/\im \delta & \bigl(H_n(X^{(n+1)},X^{(n)})=0\bigr)\\
     &= \im j_n /\im d^{CW} & (j_n\text{ injective})\\
     &= \ker \delta/ \im d^{CW} & (\text{vertical sequence exact})\\
     &= \ker d^{CW}/\im d^{CW} & (j_{n-1}\text{ injective})
   \end{align*}
 \end{proof}

 \begin{example}
   Now it is easy to compute the homology of $S^n$ for $n\ge 2$ because the cellular
   chain complex has zeros everywhere except in dimensions $n$ and $0$. Thus,
   $H_n(S^n)\cong \ZZ \cong H_0(S^n)$ and $H_k(S^n)=0$ for $k\neq 0,n$.
 \end{example}
 In general, we can compute cellular homology using the following result.
 \begin{proposition}
   If $X$ is a CW complex, let $\phi_i$ be the attaching map of the $i$-th $n$-cell, and
   let $p_j$ be the projection $X^{(n)}\to X^{(n)}/\bigl( X^{(n)}\smallsetminus e^n_j
   \bigr)$. Then the boundary map $d^{CW}:C_n(X)=\ZZ\cdot I_n\to \ZZ\cdot I_{n-1} =
   C_{n-1}(X)$ is a matrix with $(i,j)$-th entry $\deg(p_j\circ \phi_i)$.
 \end{proposition}
 \begin{proof}
   Note that we have the diagram
   \[\xymatrix{
    & H_n(X^{(n)},X^{(n-1)})\ar[r]^\delta \ar@{}[d]|{\parallel}_{\sum (\Phi_i)_* \:\wr}
    & \H_{n-1}(X^{(n-1)})\\
    \H_n(D^n)=0\ar[r]
    & \bigoplus_{I_n} H_n(D^n,S^{n-1}) \ar[r]^<>(.5)\sim_<>(.5)\delta
    & \bigoplus_{I_n} \H_{n-1}(S^{n-1}) \ar[r] \ar[u]_{\sum (\phi_i)_*} & 0=\H_{n-1}(D^n)
   }\]
   which fits (one and a half times) into the diagram
   \[\xymatrix{
   H_n(X^{(n)},X^{(n-1)}) \ar@{}[d]|{\parallel}_{\sum (\Phi_i)_*\:\wr}\ar[r]
   & \H_{n-1}(X^{(n-1)}) \ar@/_2ex/[ddr]_(.6){\sum (p_j)_*} \ar[r]
   & H_{n-1}(X^{(n-1)},X^{(n-2)}) \ar@{}[d]|{\parallel}^{\wr\: \sum (\Phi_i)_*}\\
   \ZZ\cdot I_n \ar@/_3ex/[drr]_{d^{CW}} \ar[r]_<>(.5)\delta^<>(.5)\sim & \bigoplus \H_n(S^n) \ar[u]^{\sum (\phi_i)_*}
   & \bigoplus H_{n-1}(D^{n-1},S^{n-2}) \ar[r]^<>(.5)\sim_<>(.5)\delta \ar@{}[d]|{\parallel} &
   \bigoplus \H_{n-1}(S^{n-1})\\
   & & \ZZ\cdot I_{n-1}
   }\]
   from which we see that $d^{CW} = \bigl(\sum (\phi_i)_* \bigr)\circ \bigl( \sum (p_j)_*
   \bigr)$. This proves the result.
 \end{proof}
 \begin{corollary}
   Any homology theory on CW complexes which is homotopy invariant, additive, and
   satisfies the dimension and Mayer-Vietoris axioms is isomorphic to singular homology.
 \end{corollary}
 \begin{proof}
   The results in this section show that any homology theory with these properties is
   isomorphic to \emph{its own version of CW homology}. That is, the matrix coefficients
   are the degrees of $\phi_i\circ p_j$, computed in the given homology theory. We will
   show that degree can be defined independent of homology theory.

   We have a Hurewicz map $\pi_n(S^n)\to h_n(S^n)\cong \ZZ$, which is a group
   homomorphism (using the distributivity trick that was on one of the homeworks) and
   onto because the identity map on $S^n$ gets sent to a generator of $\ZZ$. Assuming the
   Hurewicz theorem, we get that $\pi_n(S^n)\cong \ZZ$. Notice that a map $f:S^n\to S^n$
   induces a map $f_*:\pi_n(S^n)\to \pi_n(S^n)$, so we can define degree via $\pi_n$.
   Thus, degree is independent of the homology theory $h$.
 \end{proof}

 \underline{An issue you have to deal with}: If $X$ and $Y$ are CW complexes and $f:X\to
 Y$ is a map, then you don't get any natural map $H^{CW}_n(X)\to H^{CW}_n(Y)$. To get
 functoriality of $H^{CW}$, you need to prove the cellular approximation theorem (Theorem
 4.8 of Hatcher).
 \begin{theorem}
   Every map $f:X\to Y$ is homotopic to a cellular map, a map that sends $X^{(n)}$ to
   $Y^{(n)}$ for each $n$.
 \end{theorem}

 \subsection*{Morse (Floer) homology on finite-dimensional smooth manifolds}
 Choose a sufficiently nice function $f$ on a smooth manifold $M$. At a critical point of
 $f$, define the index of $f$ to be the number of negative eigenvalues of the Hessian
 (that is, the number of downward curving directions). Define $C_n^\text{Morse} =
 \ZZ\cdot I_n$, where $I_n$ is the set of index $n$ critical points of $f$. The boundary
 map counts the number of flow lines from one point to another. Now one can check that
 $d^2=0$, so we get a homology theory.

 One can check that the result is independent of the choice of $f$. In fact, $f$ dictates
 a CW structure on $M$, with one $n$-cell for each index $n$ critical point.

 In Floer homology, where $M$ is allowed to be infinite-dimensional, the function $f$ is
 somehow given to you by the geometry of the situation. You no longer get independence of
 $f$.

 \section{Homotopy invariance}
 If $f_0,f_1:X\to Y$ are homotopic maps, then we have the picture $X\xrightarrow{i_0,
 i_1} X\times I \xrightarrow{F} Y$, with $f_j=F\circ i_j$ for $j=0,1$. We would like to
 show that $f_0$ and $f_1$ induce the same map in homology. To prove that, we will
 actually show that they induce homotopic chain maps, and we will use $F$ to construct
 the homotopy. However, to do this, we must understand what the chain complex associated
 to a product looks like. This is the content of the Eilenberg-Zilber Theorem
 \ref{T:Eilenberg-Zilber}.

 \begin{definition}
   If $C_*$ and $D_*$ are two chain complexes, then their tensor product is defined by
   $(C_*\otimes D_*)_n = \bigoplus_{p+q=n} C_p\otimes_\ZZ D_q$, with boundary map $d^{C\otimes
   D}_n = \bigoplus_{p+q=n} d^C_p\otimes 1 + (-1)^p\otimes d^D_q$. The sign ensures that
   the result is a chain complex; it is a standard exercise to check this.
   \[\xymatrix{
   C_p\otimes D_q \ar[r]^{d^C\oplus 1} \ar[d]_{(-1)^p\otimes d^D} & C_{p+1}\otimes D_q \ar[d]^{(-1)^{p+1}\otimes d^D}\\
   C_p\otimes D_{q+1} \ar[r]_{d^C\otimes 1} \ar@{}[ru]|{\txt{\footnotesize anti-\\ \footnotesize commutes}} & C_{p+1}\otimes D_{q+1}
   }\]
%   \[\xymatrix @!0 @+1pc{
%     & C_{odd}\ar[r] & C_{even}\ar[r] & C_{odd}\ar[r] & C_{even} \ar[r] & \cdots \\
%   D_{even}\ar[d] & \bullet \ar[d] \ar[r]^{} & \bullet \ar[d] \ar[r] & \bullet \ar[d] \ar[r] & \bullet \ar[d]
%   \ar[r] & \cdots\\
%   D_{odd}\ar[d] & \bullet \ar[d] \ar[r] & \bullet \ar[d] \ar[r] & \bullet \ar[d] \ar[r] & \bullet \ar[d]
%   \ar[r]\ar[d] & \cdots\\
%   D_{even}\ar[d] & \bullet \ar[d] \ar[r] & \bullet \ar[d] \ar[r] & \bullet \ar[d] \ar[r] & \bullet \ar[d]
%   \ar[r] & \cdots\\
%    \vdots & \vdots & \vdots & \vdots & \vdots
%   }\]
 \end{definition}
 \begin{lemma}
   If $X$ and $X'$ are CW complexes, then $C_*(X\times X')=C(X)\otimes C(X')$
 \end{lemma}
 \begin{proof}
   This follows from the fact that $\partial (D^p\times D^q)=\partial D^p\times D^q \cup
   D^p\times \partial D^q$.
 \end{proof}
 \begin{definition}
   If $S_\bullet$ is a $\Delta$-set, the \emph{cone on $S_\bullet$} is the $\Delta$-set
   defined by $(CS)_0 = S_0\cup \infty$, $(CS)_n = S_n\cup S_{n-1}$ for $n>0$. The
   boundary maps are defined by $d_i:
   \raisebox{.5\baselineskip}{$\xymatrix @R=0pc{S_n \ar[r]^{d_i} & S_{n-1}\\
   S_{n-1} \ar[r]_{d_i} & S_{n-2}}$}$ for $i\le n$ and
   $d_{n+1}:\raisebox{.5\baselineskip}{$\xymatrix @R=0pc{S_n \ar[r]^{d_{n+1}} & S_{n-1}\\
   S_{n-1} \ar[ru]_{\id} & S_{n-2}}$}$, where $S_{-1}=\{\infty\}$.
 \end{definition}
 \anton{picture here}
 \begin{lemma}
   Cones are acyclic: $H_*\bigl(C_*(CS)\bigr)=0$.
 \end{lemma}
 \begin{proof} If $(a,b)\in \ZZ CS_n = \ZZ S_n\oplus \ZZ S_{n-1}$ maps to zero, then it
 is the image of $\bigl(0,(-1)^{n+1}a\bigr)$.
   \[\raisebox{3.5pc}{$\xymatrix{
     \ZZ S_{n+1}\ar[r]^d \ar@{}[d]|{\txt{$\oplus$}} & \ZZ S_n \ar@{}[d]|{\txt{$\oplus$}}\ar[r]^d & \ZZ S_{n-1}\ar@{}[d]|{\txt{$\oplus$}}\\
     \ZZ S_n\ar[ur]|{(-1)^{n+1}}\ar[r]^d & \ZZ S_{n-1}\ar[ur]|{(-1)^n} \ar[r]^d & \ZZ S_{n-1}
   }\qquad
   \xymatrix {
     0 \ar@{}[d]|{\txt{$\oplus$}} & a\ar@{}[d]|{\txt{$\oplus$}} \ar@{|->}[r] & da-(-1)^nb=0 \ar@{}[d]|{\txt{$\oplus$}}\\
     (-1)^{n+1}a \ar@{|->}[r]\ar@{|->}[ru] & b\ar@{|->}[r]\ar@{|->}[ru] & db=0
   }$}\qedhere\]
 \end{proof}
 \begin{corollary}[$\Delta^p\times \Delta^q$ is ``acyclic'']
   $\H_*(\Delta^p\times \Delta^q)=0$.
 \end{corollary}
 \begin{proof}
   Singular homology agrees with simplicial homology by Theorem
   \ref{T:DeltaGeomAdjunction}, so we use simplicial homology. We can break the product
   of in to simplices. Then by repeated application of Mayer-Vietoris for
   $\Delta$-sets, it is enough to show that $\H_*(\Delta^k)=0$, which is true by the lemma.
 \end{proof}

 \begin{theorem}[Eilenberg-Zilber] \label{T:Eilenberg-Zilber}
   For topological spaces $X$ and $Y$, there is a \emph{natural} chain homotopy
   equivalence $ h_{XY}:S_*X\otimes S_* Y\xrightarrow{\simeq} S_*(X\times Y)$ such that
   $(h_{XY})_0:S_0(X)\otimes S_0(Y)\to S_0(X\times Y)$ is the map $\sigma\otimes
   \sigma'\mapsto \sigma\times \sigma'$.\footnote{Note that this makes sense only because
   $\Delta^0\times \Delta^0\approx \Delta^0$.} Moreover, $h_{XY}$ is unique up to chain
   homotopy.
 \end{theorem}
 This may be rephrased as ``$S_*:(\textbf{Top},\times)\to (\textbf{Chain},\otimes)$ is a
 monoidal functor.'' We saw already that ``$S_*:(\textbf{Top},\sqcup)\to
 (\textbf{Chain},\oplus)$ is a monoidal functor.''
 \begin{proof}[Proof (via ``acyclic models'')]
   Induct on degree. The start of the induction is part of the hypothesis. Assume we have
   $h_{XY}$ defined in degree $<p+q$ for all spaces $X$ and $Y$. We would like to define
   $h_{XY}:S_p X\otimes S_q Y\to S_{p+q}(X\times Y)$. Let $a\in S_p X$ and $b\in S_q Y$,
   so $a:\Delta^p\to X$ and $b:\Delta^q\to Y$. If naturality is to hold, the following
   diagram must commute.
   \[\xymatrix @C+3pc{
   S_p X\otimes S_q Y \ar@{.>}[r]|{h_{XY}?} & S_{p+q}(X\times Y)\\
    S_p\Delta^p \otimes S_q\Delta^q \ar@{.>}[r]|{h_{\Delta^p\Delta^q}?}
   \ar[u]^{S_*a\otimes S_*b} \ar[d]_{d} & S_{p+q}(\Delta^p\times \Delta^q) \ar[u]_{S_*(a\times b)} \ar[d]^d\\
   (S_{p-1}\Delta^p\otimes S_q\Delta^q )\oplus (S_p\Delta^p\otimes S_{q-1}\Delta^q)
   \ar[r]^<>(.5){h_{\Delta^p\Delta^q}}_<>(.5){\text{induction!}} & S_{p+q-1}(\Delta^p\times \Delta^q)
   }\]
   \[\xymatrix @C+2pc{
    a\otimes b \ar@{|.>}[r] & h_{XY}(a\otimes b) \\
    \id_{\Delta^p}\otimes \id_{\Delta^q} \ar@{|->}[u]^{S_*a\otimes S_*b} \ar@{|->}[d]_d \ar@{|.>}[r]
    & y? \ar@{|->}[u]_{S_*(a\times b)} \ar@{|->}[d]^d \\
    \Bigl(\bigl(\sum_{i=0}^p (-1)^i \delta_i\bigr)\otimes \id, \id\otimes \bigl(\sum_{j=0}^q(-1)^j\delta_j\bigr)
    \Bigr) \ar@{|->}[r]^<>(.5){h_{\Delta^p\Delta^q}} & x
   }
   \]
   Now we see that a wonderful thing has happened. So long as we can find a $y\in
   S_{p+q}(\Delta^p\times \Delta^q)$ so that $dy=x$, we can define $h_{XY}(a\otimes b)$
   to be $S_*(a\times b)(y)$, and this will automatically be natural! We know that $dx=0$
   because $dx = d\circ (h_{\Delta^p\Delta^q})_{p+q-1}\circ d (\id\otimes \id) =
   (h_{\Delta^p\Delta^q})_{p+q-2}\circ d\circ d(\id\otimes \id) =0$ (note that we only
   used the degrees for which $h$ is already defined).
   Since $\Delta^p\times \Delta^q$ is acyclic $H_{p+q-1}(\Delta^p\times \Delta^q)=0$, so
   such a $y$ exists.

   It remains to show that the $h_{XY}$ only depends on the choice of $y$ up to homotopy.
   If we use $y'$ instead of $y$, we have $d(y-y')=0$. Since $H_{p+q}(\Delta^p\times
   \Delta^q)=0$, there is some $z\in S_{p+q+1}(\Delta^p\times\Delta^q)$ so that
   $dz=y-y'$. This $z$ induces a homotopy between the two choices of $h_{XY}$.
 \end{proof}
 Now we are ready to prove the desired result.
 \begin{theorem}[Homotopy invariance of Homology]
   If $f_0$ and $f_1$ are homotopic maps from $X$ to $Y$, then they induce chain
   homotopic maps $S_* X\to S_* Y$. In particular, they induce the same map on homology.
 \end{theorem}
 \begin{proof}
   Let $F:X\times I\to Y$ be a homotopy from $f_0$ to $f_1$. We have a canonical
   inclusion $C_*I\to S_*I$. Thus, we get the chain map
   \[\xymatrix @R-1pc{
    S_* X\otimes C_* I \ar[r] & S_* X\otimes S_* I\ar[r]^{\simeq}_{h_{XI}} & S_*(X\times
    I )\ar[r]^{S_*F} & S_*(Y)\\
    (S_{n+1}X\otimes C_0 I)\oplus (S_n X\otimes C_1) I \ar[rrr] & & & S_{n+1}Y
    }\]
%   \[\xymatrix{
%    (a_0\otimes v_0 + a_1\otimes v_1, b\otimes e)\ar@{|->}[r] \ar@{|->}[d]^{d^{S_*X\otimes C_*I}} &
%    S_*f_0(a_0)+S_*f_1(a_1)+h_{n+1}(b) \ar@{|->}[d]^<>(.5){d^Y}\\
%    (d^Xa_0\otimes v_0 + d^X a_1\otimes v_1 + (-1)^n b\otimes
%    (v_0-v_1), d^Xb\otimes e) \ar@{|->}[r]
%    & \raisebox{2.8pc}{\shortstack{$d^Y S_*f_0(a_0)+d^Y
%    S_*f_1(a_1) + d^Yh_{n+1}(b)$\\ $\parallel$ \\ $S_*f_0(d^X a_0) + S_*f_1(d^Xa_1) +
%    (-1)^n(S_*f_0-S_*f_1)(b) + h_n(d^Xb)$}}
%   }\]
   {\footnotesize \[\hspace*{-2pc}\xymatrix @!0 @R=3pc @C=20pc{
    (a_0\otimes v_0 + a_1\otimes v_1, b\otimes e)\ar@{|->}[r] \ar@{|->}[d]^{d^{S_*X\otimes C_*I}} &
    S_*f_0(a_0)+S_*f_1(a_1)+h_{n+1}(b) \ar@{|->}[d]^<>(.5){d^Y}\\
    (d^Xa_0\otimes v_0 + d^X a_1\otimes v_1 + (-1)^n b\otimes
    (v_0-v_1), d^Xb\otimes e) \ar@{|->}[rd] & d^Y S_*f_0(a_0)+d^Y S_*f_1(a_1) + d^Yh_{n+1}(b) \ar@{}[d]|{\parallel}\\
    & S_*f_0(d^X a_0) + S_*f_1(d^Xa_1) + (-1)^n(S_*f_0-S_*f_1)(b) + h_n(d^Xb)
   }\]}
   From the fact that it is a chain map, we get the equalities
   \begin{align*}
    \mbox{\footnotesize $d^Y S_*f_0(a_0)+d^Y S_*f_1(a_1) + d^Yh_{n+1}(b)$} &=
    \mbox{\footnotesize $S_*f_0(d^X a_0) + S_*f_1(d^Xa_1) + (-1)^n(S_*f_0-S_*f_1)(b) + h_n(d^Xb)$}\\
    d^Y h_{n+1} &= (-1)^n(S_*f_0-S_*f_1) + h_nd^X & \llap{($d^Y\circ S_*f_i= S_*f_i\circ d^Y$)}\\
    d^Y \circ (-1)^n h_{n+1} + (-1)^{n-1} h_n \circ d^X &= S_*f_0-S_*f_1
   \end{align*}
   Thus, if we twist $h$ by some minus signs as above, we get a homotopy from $S_*f_0$ to
   $S_*f_1$, as desired.
 \end{proof}

 \section{The Generalized Jordan Curve Theorem}
 The following theorem looks obvious until you think about it.
 \begin{theorem}[Jordan Curve Theorem]
   If $S^1\hookrightarrow \RR^2$ is an embedding (continuous injection), the the
   complement of the image has two path components.
 \end{theorem}
 We will prove a more general version.
 \begin{theorem} \label{T:GJCT}
   If $i:S^r\hookrightarrow S^n$ with $r<n$, then
   $\H_k\bigl(S^n\smallsetminus i(S^r)\bigr)=\begin{cases}
     \ZZ & k=n-r-1\\
     0 & \text{else}
   \end{cases}.$
 \end{theorem}
 That is, homology is independent of the embedding. This is kind of sad because it tells
 us that we cannot use homology to understand knots.

 You can guess the answer by considering linear embeddings, where we think of $S^n$ as
 $\RR^n\cup \infty$. Then we have
 \begin{align*}
   S^n\smallsetminus S^r &= \RR^n\smallsetminus \RR^r\\
    &= (\RR^{n-r} \times \RR^r) \smallsetminus (0\times \RR^r)\\
    &= (\RR^{n-r}\smallsetminus 0)\times \RR^r \simeq S^{n-r-1}.
 \end{align*}
 However, you have to be careful. It is not true that for any embedding $i$,
 $S^n\smallsetminus i(S^r)\simeq S^{n-r-1}$. For example, we have the Alexander horned
 sphere in $S^3$, whose complement is not homotopic to $S^0$.

 For the proof of Theorem \ref{T:GJCT}, we'll need the following lemma.
 \begin{lemma}
   Let $Y$ be compact and assume it has the property that for every $i:Y\hookrightarrow
   S^n$, $\H\bigl(S^n\smallsetminus i(Y)\bigr)=0$. Then $Y\times I$ has the same
   property.
 \end{lemma}
 \begin{proof}
   Let $f:Y\times I\hookrightarrow S^n$ be an embedding. Let $U_0=S^n\smallsetminus
   f(Y\times [0,1/2])$ and let $U_1=S^n\smallsetminus f(Y\times [1/2,1])$. Then $U_0\cup
   U_1 = S^n \smallsetminus f(Y\times \{1/2\})$, so it has trivial homology by
   assumption, and $U_0\cup U_1 = S^n\smallsetminus f(Y\times I)$. By Mayer-Vietoris, we
   get the isomorphism
   \[
    \H_k\bigl(S^n\smallsetminus f(Y\times I)\bigr) \xrightarrow{\ \sim\ } \H_k(U_0)\oplus
    \H_k(U_1).
   \]
   Any non-zero homology class would have to give a non-zero homology class in one of the
   $U_i$. Then we can induct to get non-zero homology classes in $S^n\smallsetminus
   f(Y\times [p-\varepsilon,p+\varepsilon])$ for some $p\in I$ and arbitrarily small
   $\varepsilon$. Since homology commutes with filtered colimits, we get a non-zero
   homology class in $S^n\smallsetminus f(Y\times\{p\})$, a contradiction.
 \end{proof}
 In particular, $D^0=pt$ has the property in the lemma because $S^n\smallsetminus pt\cong
 \RR^n\simeq pt$. By the lemma, $D^r\cong D^{r-1}\times I$ has the property for all $r$.
 This seems obvious, but you have to watch out, it is not true that $S^n\smallsetminus
 i(D^r)\simeq pt$ for all embeddings $i$. For example, we have the Fox-Artin wild arc in
 $S^3$.
 \begin{proof}[Proof of Theorem \ref{T:GJCT}]
   Induct on $r$. For $r=0$, our na\"\i ve calculation works; $S^n\smallsetminus
   i(S^0)\simeq S^{n-1}$ for all $i$. Now assume the result for $r-1$, and let
   $i:S^r\hookrightarrow S^n$. Then we can write $S^r = D^r_+\cup_{S^{r-1}} D^r_-$, and
   we have open sets $U= S^n\smallsetminus i(D^r_+)$ and $V=S^n\smallsetminus i(D^r_-)$
   in $S^n$, with
   \begin{align*}
     S^n\smallsetminus i(S^{r-1}) &= U\cup V\\
     S^n\smallsetminus i(S^r) &= U\cap V.
   \end{align*}
   By the Lemma, $U$ and $V$ have trivial homology. Mayer-Vietoris and the induction step
   immediately give the desired result.
 \end{proof}
 There are some related results.
 \begin{theorem}[Schoenflie\ss\ Conjecture, proven by Mazur and Brown]
   If $i:S^{n-1}\times (-\varepsilon,\varepsilon)\hookrightarrow S^n$, then
   $S^n\smallsetminus i\bigl( S^{n-1}\times (-\varepsilon,\varepsilon)\bigr) \approx
   D^n\sqcup D^n$. For $n=2$, you don't even need the collar neighborhood.
 \end{theorem}
 \begin{theorem}[Annulus Conjecture, proven by Kirby for $n\ge 5$, Quinn for $n=4$, and somebody for $n\le 3$]
   If $i:\bigl(S^{n-1}\times (-\varepsilon,\varepsilon)\bigr) \sqcup (S^{n-1}\times
   (-\varepsilon,\varepsilon)\bigr) \hookrightarrow S^n$, then the (appropriate component
   of) the complement of $\im i$ is homeomorphic to $S^{n-1}\times I$.
 \end{theorem}

 \section{Lefshetz, Alexander, and Poincar\'e Dualities}

 \begin{definition}
   Let $H_*$ be a homology theory obtained from a chain complex, $H_n:\textbf{hTop}\to
   \textbf{Chain}\xrightarrow{n\text{-th homology}} \textbf{Ab}$. Then we define the
   corresponding \emph{cohomology} theory by dualizing the chain complex in the middle,
   $H^n:\textbf{hTop}\to \textbf{Chain}\xrightarrow{\hom(-,\ZZ)}
   \textbf{Chain}\xrightarrow{n\text{-th homology}} \textbf{Ab}$.
 \end{definition}
 In homework 13, we showed uniqueness of cohomology.

 \begin{theorem}[Lefschetz Duality]
   If $M$ is an $n$-dimensional compact orientable manifold, then $H_k(M,\partial M)\cong
   H^{n-k}(M)$ and $H_k(M)\cong H^{n-k}(M,\partial M)$.
 \end{theorem}

 \begin{theorem}[Alexander Duality]
   If $K$ is a finite CW complex and $i:K\hookrightarrow \RR^n$ is an embedding, then
   $\H_k\bigl(\RR^n\smallsetminus i(K)\bigr)\cong H^{n-k-1}(K)$.
 \end{theorem}
 \begin{remark}[Excision axiom]
   Originally, the axioms of homology were stated for pairs of spaces, and one of the
   axioms was \emph{excision}: if $\bbar B\subseteq\inn A$, then $H_n(X,A)\cong
   H_n(X\smallsetminus B,A\smallsetminus B)$. It is easy to check that this holds in our
   formulation of homology. We define $H_n(X,A)$ as $\H_n(X/A)$ \anton{at least when
   $A\subseteq X$ is a cofibration ... what if it isn't?}
 \end{remark}
 \begin{proof}
   There is a regular $\varepsilon$-neighborhood $N(K)$ of $i(K)$. That is, there is a
   compact $n$-manifold $N(K)$ which retracts to $i(K)$. $N(K)$ is oriented because it is
   an $n$-submanifold in the oriented $n$-manifold $\RR^n$. Let $N'(K)$ be an
   $\varepsilon/2$-neighborhood of $i(K)$. Now we can compute
   \begin{align*}
     H^{n-k-1}(K) &\cong H^{n-k-1}\bigl(N(K)\bigr)\\
        &\cong H_{k+1}\bigl(N(K),\partial N(K)\bigr) & \text{(Lefschetz)}\\
        &\cong H_{k+1}\bigl(\RR^n,\RR^n\smallsetminus \inn N'(K)\bigr) &\text{(excision)}\\
        &\cong H_{k+1}\bigl(\RR^n,\RR^n\smallsetminus i(K)\bigr)\\
        &\cong \H_k\bigl(\RR^n\smallsetminus i(K)\bigr)
   \end{align*}
   where the last isomorphism follows from the long exact reduced pair sequence and the
   fact that $\H_*(\RR^n)=0$.
 \end{proof}
 Note that we can now compute $H_k\bigl(S^n\smallsetminus i(K)\bigr)$. In particular,
 the Generalized Jordan Curve Theorem is a corollary of Alexander Duality. Use
 Mayer-Vietoris, with $U$ an neighborhood of infinity that doesn't intersect $i(K)$, and
 $V$ a big ball containing the (missing) image of $K$. Then we have that $U\simeq \ast$,
 $V\simeq \RR^n\smallsetminus i(K)$, and $U\cap V\simeq S^{n-1}$.
 \[
    H_k(S^{n-1})\to H_k\bigl(\RR^n\smallsetminus i(K)\bigr) \to
    H_k\bigl(S^n\smallsetminus i(K)\bigr) \to H_{k-1}(S^{n-1})
 \]
 If $k\neq n,n-1$, then we get $H_k\bigl(\RR^n\smallsetminus i(K)\bigr) \cong
 H_k\bigl(S^n\smallsetminus i(K)\bigr)$. For the remaining cases, we use naturality of
 Mayer-Vietoris. We have inclusions $\RR^n\smallsetminus i(K)\hookrightarrow
 \RR^n\smallsetminus pt$ and $S^n\smallsetminus i(K)\hookrightarrow S^n\smallsetminus pt$.
 The corresponding map on $S^{n-1}$ is the identity. Thus, we get
 {\small
 \[\hspace*{-3em}
 \xymatrix @C-.8pc{
    0\ar[r] & \H_n\bigl(\RR^n\smallsetminus i(K)\bigr) \ar[r] \ar[d]
    & \H_n\bigl(S^n\smallsetminus i(K)\bigr) \ar[r]^<>(.5)b \ar[d]
    & \H_{n-1}(S^{n-1}) \ar[r]^<>(.5)a \ar[d]^{\wr}
    & \H_{n-1}\bigl(\RR^n\smallsetminus i(K)\bigr) \ar[r] \ar[d]^{j}
    & \H_{n-1}\bigl(S^n\smallsetminus i(K)\bigr) \ar[r] \ar[d]
    & 0\\
    0\ar[r] & \H_n(\RR^n\smallsetminus pt) \ar[r]
    & \H_n(S^n\smallsetminus pt) \ar[r]
    & \H_{n-1}(S^{n-1}) \ar[r]^<>(.5){\sim}
    & \H_{n-1}(\RR^n\smallsetminus pt) \ar[r]
    & \H_{n-1}(S^n\smallsetminus pt) \ar[r]
    & 0\\
 }\]} We can deduce that $a$ must be an injection (since $j\circ a$ is an isomorphism),
 so $b$ must be the zero map. In fact, one can show that $a$ splits. Thus,
 \[
   \H_k\bigl(\RR^n\smallsetminus i(K)\bigr) \cong \begin{cases}
   \H_k\bigl(S^n\smallsetminus i(K)\bigr) & k\neq n-1\\
   \H_k\bigl(S^n\smallsetminus i(K)\bigr)\oplus \ZZ & k= n-1
 \end{cases}\]
 and the extra factor of $\ZZ$ is represented by a big sphere near infinity.

 Another corollary of Lefshetz duality is Poincar\'e duality. The proof is immediate.
 \begin{theorem}[Poincar\'e Duality]
   If $M$ is a closed orientable $n$-manifold, then $H_k(M)\cong H^{n-k}(M)$.
 \end{theorem}
 Note that Poincar\'e duality implies that $H^r(M)=0=H_r(M)$ for $r$ larger than the
 dimension of $M$.

 The idea of Poincar\'e duality is this. Assume $M=|S_\bullet|$ for some $S_\bullet$ (in
 fact, not every manifold is triangulatable, so this is a non-trivial assumption). Then
 $H_*(M)=H_*(S_\bullet)$. We get a cellular chain $C_*(S_\bullet)$, and we can construct
 $M$ as the dual CW complex, in which every $k$-cell is replaced by an
 $(n-k)$-cell.\footnote{In the case where $M$ is a surface, so the CW structure is a
 graph on $M$, the dual decomposition is the usual dual graph.} It is easy to see that
 $C_{k}\cong \hom(C_{n-k}^\text{dual},\ZZ)$. Thus, taking homologies, we get $H_k(M)\cong
 H^{n-k}(M)$.

 \section{Coefficients and the K\"unneth Theorem}

 \begin{definition}
   Homology (resp.~cohomology) with coefficients in some abelian group $A$ is defined by
   the Eilenberg-Steenrod axioms, with the dimension axiom replaced by
   $H_n(pt;A)=\delta_{n0} A$ (resp.~$H^n(pt;A)=\delta_{n0}A$).
 \end{definition}
 The proofs of uniqueness work just fine. For singular homology, we define
 $H_n(X;A)$ as the $n$-th homology of $S_*(X)\otimes_\ZZ A$ and $H^n(X;A)$ as the $n$-th
 homology of $\hom_\ZZ\bigl(S_*(X),A\bigr)$. The replacement of the Hurewicz map is a map
 $\pi_n(X)\otimes_\ZZ A\to H_n(X;A)$.

 Note that we cannot define $H_n(X;A)$ as $H_n(X)\otimes_\ZZ A$ because $\otimes_\ZZ A$
 is not exact, so the Mayer-Vietoris axiom would not hold. However, the Universal
 Coefficients Theorem says we wouldn't be wrong by too much.

 \begin{definition}
   Given two complexes $C_*$ and $D_*$, we have the inclusion $C_p\otimes D_q\to
   (C_*\otimes D_*)_{p+q}$. Because of how the boundary maps in a tensor product are
   defined, this induces a map $H_p(C_*)\otimes H_q(D_*)\xrightarrow{\ \times\ }
   H_{p+q}(C_*\otimes D_*)$.

   The \emph{cross product} in homology is the map $H_p(X)\otimes
   H_q(Y)\xrightarrow{\ \times\ } H_{p+q}(X\times Y)$. The cross product in
   cohomology is the map $H^p(X)\otimes H^q(Y)\xrightarrow{\ \times\ } H^{p+q}(X\times Y)$.
 \end{definition}

 \begin{theorem}[Algebraic K\"unneth Theorem]
   If $C_*$ and $D_*$ are free chain complexes over a PID $R$, then for each $n$ there is
   a natural short exact sequence
   \[
    0\to \bigoplus_{p+q=n} \Bigl( H_p(C_*)\otimes_R H_q(D_*) \Bigr) \xrightarrow{\
    \times\ } H_n(C_*\otimes_R D_*) \to \bigoplus_{p+q=n-1}
    \tor_R\bigl(H_p(C_*),H_q(D_*)\bigr) \to 0.
   \]
   This sequence splits, but not naturally.
 \end{theorem}
 This is Theorem 3B.5 of Hatcher. \anton{If $R$ is not a PID, then there are higher
 $\tor$ groups} Properties of $\tor_R$:
 \begin{enumerate}
   \item $\tor_R(M,N)=\tor_R(N,M)$
   \item $\tor_R(\bigoplus_i M_i,N)\cong \bigoplus_i \tor_R(M_i,N)$
   \item $\tor_R(M,N)=\tor_R\bigl( T(M),N\bigr)$, where $T(M)$ is the torsion part of $M$
   \item $\tor_R(R/I,N) = \ker (I\otimes_R N\to R\otimes_R N)$
 \end{enumerate}

 \begin{corollary}[Universal Coefficients for Homology]
   There is a natural exact sequence (which splits un-naturally)
   \[
    0\to H_n(X)\otimes A\to H_n(X;A)\to \tor_\ZZ\bigl(H_{n-1}(X),A\bigr)\to 0.
   \]
 \end{corollary}
 Just take $A$ to be a chain complex concentrated in degree 0 and apply the K\"unneth
 Theorem.

 \begin{corollary}[Topological K\"unneth Theorem]
   There is a natural short exact sequence which splits un-naturally
   \[
    0\to \bigoplus_{p+q=n} \bigl( H_p(X)\otimes H_q(Y) \bigr) \xrightarrow{\ \times\ }
    H_n(X\times Y) \to \bigoplus_{p+q=n-1} \tor \bigl(H_p(X),H_q(Y)\bigr) \to 0.
   \]
 \end{corollary}

 \begin{theorem}[Universal Coefficients for Cohomology]
   There is a natural short exact sequence which splits un-naturally
   \[
    0\to \ext_\ZZ \bigl(H_{n-1}(X),A\bigr)\to H^n(X;A)\to \hom_\ZZ\bigl(H_n(X),A\bigr)\to
    0.
   \]
 \end{theorem}
 \anton{maybe there should be something about the algebraic Dual K\"unneth theorem here}
 Properties of $\ext$:
 \begin{enumerate}
   \item $\ext (\bigoplus_i A_i,B)\cong \prod_i \ext(A_i,C)$
   \item $\ext (A,\prod_i B_i) \cong \prod_i \ext(A,B_i)$
   \item $\ext (A,B)=0$ if $A$ is free
   \item $\ext (\ZZ/n,\ZZ)\cong \ZZ/n$
   \item $\ext (\ZZ/n,\ZZ/m) \cong \ZZ/gcd(m,n)$
 \end{enumerate}


 \anton{the rest of this section needs to be organized}
 \begin{definition} \label{Def:orient}
   \anton{What is the definition of $A$-orientable for an abelian group $A$?}
 \end{definition}
 \begin{definition}
   An orientation of a real finite-dimensional vector space $V$ is an ordered basis up to
   positive determinant change. This is the same as a generator for
   $H_n(V,V\smallsetminus 0) \cong \H_{n-1}(V\smallsetminus 0)\cong \ZZ$.
 \end{definition}
 \begin{definition}
   An orientation of a manifold $M^n$ is a compatible choice of generators $\mu_x\in
   H_n(M,M\smallsetminus x)\cong \ZZ$ for all points $x\in M$. Compatible means that
   there is an open cover so that $\mu_U\in H_n(M,M\smallsetminus U)$ restricts to
   $\mu_x$ for all $x\in U$.
 \end{definition}
 \begin{remark}
   Any manifold is $\ZZ/2\ZZ$-oriented, so we can apply all the duality theorems to get
   information, so long as we use $\ZZ/2\ZZ$ coefficients.
 \end{remark}

 \section{Cup product in Cohomology}
 \begin{definition}
   The cup product in cohomology is defined as $\smile:H^p(X)\otimes H^q(X)\xrightarrow{\
   \times\ } H^{p+q}(X\times X)\xrightarrow{\Delta^*} H^{p+q}(X)$, where $\Delta:X\to
   X\times X$ is the diagonal map.
 \end{definition}
 \begin{theorem}
   $H^*(X)=\bigoplus H^p(X)$ is naturally a graded commutative ring under cup product.
   Recall that commutativity for a graded ring means $a\smile b=(-1)^{|a|\cdot
   |b|}b\smile a$.
 \end{theorem}
 \begin{remark}\label{Rmk:H^*(XxY)}
   If the $\tor$ terms in the K\"unneth formula are zero, then $\times$ is an
   isomorphism, so we get coring structure on $H_*(X)$.
 \end{remark}
 \begin{remark}
   If $a\in H^p(X)$ and $b\in H^q(Y)$, then $a\times b=p_1^*a\smile p_2^* b$, where
   $\times$ is the cross product, and $p_1$ and $p_2$ are the projections from $X\times
   Y$ to $X$ and $Y$, respectively. This is a handy way to compute cup products.

   If some $\tor$ terms are zero \anton{which?}, then $H^*(X\times Y)\cong H^*(X)\otimes
   H^*(Y)$ as a graded commutative ring, where we define $(a\otimes b)\cdot (c\otimes
   d)=(-1)^{|b|\cdot |c|}ac\otimes bd$.
 \end{remark}
 \begin{remark}
   The 1 in the ring $H^*(X)$ is given by the natural map $\ZZ\cong H^*(pt)\to H^*(X)$.
   In particular, since $H^*(pt)$ is concentrated in degree 0, the 1 in $H^*(X)$ is in
   degree zero.
 \end{remark}
 \begin{example}
   $H^*(S^1)\cong \ZZ\oplus x\ZZ \cong \ZZ[x]/(x^2)$, where $x\in H^1(S^1)$ is a
   generator. Note that $x\smile x=0$ because it must lie in degree $2$.

   More generally, $H^*(S^n)\cong \ZZ[x]/(x^2)$, where $|x|=n$.
 \end{example}
 \begin{example}
   $H^*(S^n\times S^m) \cong \ZZ[x,y]/(x^2,y^2)$, with $|x|=n$ and $|y|=m$. This follows
   from Remark \ref{Rmk:H^*(XxY)}.
 \end{example}
 \begin{example}
   What is the ring $H^*(\CC\PP^2)$? We claim it is $\ZZ[z]/(z^3)$, with $z\in
   H^2(\CC\PP^2)\cong \ZZ$. \anton{how does one prove this?}
 \end{example}

\end{document}
